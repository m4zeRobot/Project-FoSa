% ### Erste Seite der Potentialtheorie ###
\begin{center}
    \section*{\underline{Potentialtheorie}}
\end{center}

\begin{tikzpicture}[remember picture, overlay]

% --- ERSTE REIHE: Am Seitenursprung anfangen
\setlength{\rowheight}{20mm}

    % A1) Erste Kachel relativ zum Seitenursprung
    \setlength{\cellwidth}{70mm}
    \Tile{A1}{([xshift=10mm, yshift=-20mm]current page.north west)}{\cellwidth}{\rowheight}
        {
            \vspace{-2mm}
            \titlebf{Geschwindigkeitspotential} \\[1.5ex]
            \begin{tabularx}{\dimexpr\cellwidth-2\TileSep}{Z|Z}
                \begin{tabular}[c]{@{}c@{}}
                    \myul{Definition} \\[2.0ex]
                    $\underline{v} = \nabla(\phi)$ \vspace{2mm}
                \end{tabular} & \begin{tabular}[c]{@{}c@{}}
                    \myul{Inkompressibel} \\[2.0ex]
                    $\Delta(\phi) = 0$ \vspace{2mm}
                \end{tabular}
            \end{tabularx}
        }{0}{1}{1}{0}

    % B1) Zweite Kachel relativ zu A1
    \setlength{\cellwidth}{62mm}
    \Tile{B1}{(A1.north east)}{\cellwidth}{\rowheight}
        {
            \vspace{-2mm}
            \titlebf{Grundgleichung} \\[-1.5ex]
            \begin{equation*}
                \frac{\partial\phi}{\partial t} +\frac{1}{2} \left(\nabla(\phi)\right)^2 + \frac{r}{\rho} + U = F(t)
            \end{equation*}
        }{0}{1}{1}{0}

    % C1) Zweite Kachel relativ zu B1
    \setlength{\cellwidth}{58mm}
    \Tile{C1}{(B1.north east)}{\cellwidth}{\rowheight}
        {
            \vspace{-2mm}
            \titlebf{Zirkulation} \\[-1.5ex]
            \begin{equation*}
                \Gamma_{1\rightarrow 2} = \int_1^2 \underline{v}\cdot\text{d}\underline{s} = \phi_2 - \phi_1
            \end{equation*}
        }{0}{0}{1}{0}

% --- Zweite REIHE: direkt an erster Reihe
\setlength{\rowheight}{26mm}

    % A2) Erste Kachel relativ zu A1
    \setlength{\cellwidth}{98mm}
    \Tile{A2}{(A1.south west)}{\cellwidth}{\rowheight}
        {
            \titlebf{Laplace-Gleichung (ausgeschrieben)} \\[1.5ex]
            \begin{tabularx}{\dimexpr\cellwidth-2\TileSep}{c|c}
                \begin{tabular}[c]{@{}c@{}}
                    \myul{Kartesisch} \\[2.0ex]
                    $\dfrac{\partial^2 \phi}{\partial x^2} + \dfrac{\partial^2 \phi}{\partial y^2} + \dfrac{\partial^2 \phi}{\partial z^2} = 0$ \vspace{2mm}
                \end{tabular}\hspace{1mm} & \begin{tabular}[c]{@{}c@{}}
                    \myul{Zylinderkoordinaten} \\[2.0ex]
                    \hspace{2mm}$\dfrac{1}{r} \dfrac{\partial}{\partial r} \left( r \dfrac{\partial \phi}{\partial r} \right) + \dfrac{1}{r^2} \dfrac{\partial^2 \phi}{\partial \varphi^2} + \dfrac{\partial^2 \phi}{\partial z^2} = 0$ \vspace{2mm}
                \end{tabular}
            \end{tabularx}
        }{0}{1}{1}{0}

    % B2) Zweite Kachel relativ zu A2
    \setlength{\cellwidth}{92mm}
    \Tile{B2}{(A2.north east)}{\cellwidth}{\rowheight}
        {
            \titlebf{Randbedingungen} \\[1.5ex]
            \begin{tabularx}{\dimexpr\cellwidth-2\TileSep}{Z|Z|Z}
                \begin{tabular}[c]{@{}c@{}}
                    \myul{Feste Wand} \\[2.0ex]
                    $\dfrac{\partial\phi}{\partial n} = 0$ \vspace{3.8mm}
                \end{tabular} & \begin{tabular}[c]{@{}c@{}}
                    \myul{Bewegte Wand} \\[2.0ex]
                    $\dfrac{\partial\phi}{\partial n} = \underline{u}\cdot\underline{n}$ \vspace{3.8mm}
                \end{tabular} & \begin{tabular}[c]{@{}c@{}}
                    \myul{Im Unendlichen} \\[2.0ex]
                    $\nabla(\phi)=\underline{v}_\infty$ \vspace{3.8mm}
                \end{tabular}
            \end{tabularx}
        }{0}{0}{1}{0}

% --- Dritte REIHE: unter A2, dann rechts anbauen
\setlength{\rowheight}{36mm}

    % A3) Erste Kachel relativ zu A2
    \setlength{\cellwidth}{95mm}
    \Tile{A3}{(A2.south west)}{\cellwidth}{\rowheight}
        {
            \titlebf{Ebene Geschwindigkeiten} \\[1.5ex]
            \begin{tabularx}{\dimexpr\cellwidth-2\TileSep}{Z|Z}
                \begin{tabular}[c]{@{}c@{}}
                    \myul{Kartesisch} \\[2.5ex]
                    $u = \dfrac{\partial\phi}{\partial x} = \dfrac{\partial\psi}{\partial y}$ \\[3.0ex]
                    $v = \dfrac{\partial\phi}{\partial y} = -\dfrac{\partial\psi}{\partial x}$ \vspace*{2mm}
                \end{tabular} & \begin{tabular}[c]{@{}c@{}}
                    \myul{Kartesisch} \\[2.5ex]
                    $u_r = \dfrac{\partial\phi}{\partial r} = \dfrac{1}{r}\cdot\dfrac{\partial\psi}{\partial\varphi}$ \\[3.0ex]
                    $v_\varphi = \dfrac{1}{r}\cdot\dfrac{\partial\phi}{\partial\varphi} = -\dfrac{\partial\psi}{\partial r}$ \vspace*{2mm}
                \end{tabular}
            \end{tabularx}
        }{0}{1}{1}{0}

    % B3) Zweite Kachel relativ zu A3
    \setlength{\cellwidth}{95mm}
    \Tile{B3}{(A3.north east)}{\cellwidth}{\rowheight}
        {
            \titlebf{Geschwindigkeitsbeziehungen} \\[1.5ex]
            \begin{tabularx}{\dimexpr\cellwidth-2\TileSep}{Z|Z}
                \begin{tabular}[c]{@{}c@{}}
                    \myul{Eben (Cauchy-Riemann)} \\[2.5ex]
                    $u = \dfrac{\partial\phi}{\partial x} = \dfrac{\partial\psi}{\partial y}$ \\[3.0ex]
                    $v = \dfrac{\partial\phi}{\partial y} = -\dfrac{\partial\psi}{\partial x}$ \vspace*{2mm}
                \end{tabular} & \begin{tabular}[c]{@{}c@{}}
                    \myul{Rotationssymmetrisch} \\[2.5ex]
                    $v_r = \dfrac{\partial\phi}{\partial r} = -\dfrac{1}{r}\cdot\dfrac{\partial\psi}{\partial z}$ \\[3.0ex]
                    $v_z = \dfrac{\partial\phi}{\partial z} = \dfrac{1}{r}\cdot\dfrac{\partial\psi}{\partial r}$ \vspace*{2mm}
                \end{tabular}
            \end{tabularx}
        }{0}{0}{1}{0}

% --- Vierte REIHE: unter A3, dann rechts anbauen
\setlength{\rowheight}{40mm}

    % A4) Erste Kachel relativ zu A3
    \setlength{\cellwidth}{75.5mm}
    \Tile{A4}{(A3.south west)}{\cellwidth}{\rowheight}
        {
            \titlebf{Resultierende Geschw.}
            \begin{equation*}
                F(z) = F(x+iy) = \phi(x,y)+i\psi(x,y)
            \end{equation*} \\[-1.5ex]
            \rule{\dimexpr\cellwidth-2\TileSep}{0.5pt} \\
            \vspace{1mm}
            \begin{tabularx}{\dimexpr\cellwidth-2\TileSep}{Z|Z}
                \begin{tabular}[c]{@{}c@{}}
                    \myul{Geschwindigkeiten} \\[1.5ex]
                    $w = u + iv$ \\[1.5ex]
                    $w_* = u - iv = \dfrac{\text{d}F(z)}{\text{d}z}$ \vspace*{2mm}
                \end{tabular} & \hspace{1mm}\begin{tabular}[c]{@{}c@{}}
                    \myul{Druckverteilung} \\[3.5ex]
                    $p = p_0 - \dfrac{\rho}{2} a^2 r^{2(n-1)}$ \vspace*{6mm}
                \end{tabular}
            \end{tabularx}
        }{0}{1}{1}{0}

    % B4) Zweite Kachel relativ zu A4
    \setlength{\cellwidth}{54.5mm}
    \Tile{B4}{(A4.north east)}{\cellwidth}{\rowheight}
        {
            \titlebf{Ebene Translationsströmung} \\[-1.5ex]
            \begin{equation*}
                F(z) = a\cdot z = (a_1 - i a_2)\cdot z
            \end{equation*}\\[-4.0ex]
            \begin{align*}
                \phi &= a_1 \cdot x + a_2\cdot y \\
                \psi &= a_1 \cdot y - a_2\cdot x
            \end{align*}
            \begin{equation*}
                u=a_1\ ;\ \ \ v=a_2
            \end{equation*}
        }{0}{1}{1}{0}

    % C4) Dritte Kachel relativ zu B4
    \setlength{\cellwidth}{60mm}
    \Tile{C4}{(B4.north east)}{\cellwidth}{\rowheight}
        {
            \titlebf{Ebene Eckenströmung} \\[-2.0ex]
            \begin{equation*}
                F(z) = \frac{a}{n}z^n = \frac{a}{n}r^n (\cos(n\varphi)+i\sin(n\varphi))
            \end{equation*}\\[-4.0ex]
            \begin{align*}
                \phi &= \tfrac{a}{n}r^n\cos(n\varphi) \\[1.0ex]
                \psi &= \tfrac{a}{n}r^n\sin(n\varphi)
            \end{align*}
            \begin{equation*}
                |w| = \left|a\cdot z^{n-1}\right| = |a|\cdot r^{n-1}
            \end{equation*}
        }{0}{0}{1}{0}

% --- Fünfte REIHE: unter A4, dann rechts anbauen
\setlength{\rowheight}{58mm}

    % A5) Erste Kachel relativ zu A4
    \setlength{\cellwidth}{35mm}
    \Tile{A5}{(A4.south west)}{\cellwidth}{\rowheight}
        {
            \titlebf{Feste Wände} \\
            \vspace*{4mm}
            \begin{equation*}
                \varphi_{k,n} = k\cdot\dfrac{\pi}{n}
            \end{equation*}
            \begin{equation*}
                \varepsilon = \dfrac{\pi}{n}
            \end{equation*}
            \begin{equation*}
                \vartheta = \pi - \varepsilon = \pi\cdot\dfrac{n-1}{n}
            \end{equation*}
        }{0}{1}{1}{0}

    % B5) Zweite Kachel relativ zu A5
    \setlength{\cellwidth}{52mm}
    \Tile{B5}{(A5.north east)}{\cellwidth}{\rowheight}
        {
            \titlebf{$\sfrac{1}{2} < n < 1$} \\
            \vspace*{2mm}
            \includegraphics[width=34mm]{images/konvexe_ecke.pdf} \\
            \vspace*{2mm}
            Konvexe Ecke
            \begin{equation*}
                \left|w\right|\sim\dfrac{1}{r^{n-1}}
            \end{equation*}
            $r = 0 \Rightarrow$ Eckenstr. mit $\left|w\right|\rightarrow\infty$
        }{0}{1}{1}{0}

    % C5) Dritte Kachel relativ zu B5
    \setlength{\cellwidth}{46mm}
    \Tile{C5}{(B5.north east)}{\cellwidth}{\rowheight}
        {
            \titlebf{$1 < n < 2$} \\
            \vspace*{4mm}
            \includegraphics[width=39.5mm]{images/konkave_ecke.pdf} \\
            \vspace*{3mm}
            Konkave Ecke \\[3.0ex]
            $\left|w\right|\sim r^{n-1}$ \\[3.0ex]
            $r = 0 \Rightarrow$ Staupunkt
        }{0}{1}{1}{0}

    % D5) Vierte Kachel relativ zu C5
    \setlength{\cellwidth}{57mm}
    \Tile{D5}{(C5.north east)}{\cellwidth}{\rowheight}
        {
            \titlebf{$n > 2$} \\
            \vspace*{5mm}
            \includegraphics[width=49mm]{images/scharfe_konkave_ecke.pdf} \\
        }{0}{0}{1}{0}

% --- Sechse REIHE: unter A5, dann rechts anbauen
\setlength{\rowheight}{55mm}

    % A6) Erste Kachel relativ zu A5
    \setlength{\cellwidth}{78mm}
    \Tile{A6}{(A5.south west)}{\cellwidth}{\rowheight}
        {
            \titlebf{$n = \sfrac{1}{2}$} \\[-3.0ex]
            \begin{equation*}
                F(z) = 2\cdot a \cdot \sqrt{z} = 2\cdot a \cdot \sqrt{r}\cdot \left( \cos\left(\frac{\varphi}{2}\right) + i \sin\left(\frac{\varphi}{2}\right) \right)
            \end{equation*}
            \includegraphics[height=28mm]{images/ebene-randumströmung.pdf} \\[1.0ex]
            Ebene Randumströmung
        }{0}{1}{1}{0}

    % B6) Zweite Kachel relativ zu A6
    \setlength{\cellwidth}{56mm}
    \Tile{B6}{(A6.north east)}{\cellwidth}{\rowheight}
        {
            \titlebf{$n = 1$} \\[-3.0ex]
            \begin{equation*}
                F(z) = a \cdot z
            \end{equation*}
            \includegraphics[height=28mm]{images/ebene-translationsströmung.pdf} \\[2.5ex]
            Ebene Translationsströmung
        }{0}{1}{1}{0}

    % C6) Dritte Kachel relativ zu B6
    \setlength{\cellwidth}{56mm}
    \Tile{C6}{(B6.north east)}{\cellwidth}{\rowheight}
        {
            \titlebf{$n = 2$} \\[-3.0ex]
            \begin{equation*}
                F(z) = \frac{a}{2} \cdot z^2
            \end{equation*}
            \includegraphics[height=28mm]{images/ebene-staupunktströmung.pdf} \\[1.5ex]
            Ebene Staupunktströmung
        }{0}{0}{1}{0}

% --- Siebte REIHE: unter A6, dann rechts anbauen
\setlength{\rowheight}{45mm}

    % A7) Erste Kachel relativ zu A6
    \setlength{\cellwidth}{190mm}
    \Tile{A7}{(A6.south west)}{\cellwidth}{\rowheight}
        {
            \titlebf{Strömungsbild bestimmen} \\[3.0ex]

            \begin{tikzpicture}[node distance=6mm, >=Stealth]
                \node[draw, rounded corners, minimum width=1cm, minimum height=8mm] (A) {
                    $\varPsi(x,y)$
                };
                \node[draw, rounded corners, minimum width=1cm, minimum height=8mm, right=of A] (B) {
                    $u(x,y), v(x,y)$
                };
                \node[draw, rounded corners, minimum width=1cm, minimum height=8mm, right=of B] (C) {
                $
                    \left.
                    \begin{aligned}
                        u &= 0 \\
                        v &= 0
                    \end{aligned}
                    \ \right\}\Rightarrow\text{ Staupunkte }x_{0,i}\text{, }y_{0,i}
                $
                };
                \node[draw, rounded corners, minimum width=1cm, minimum height=8mm, right=of C] (D) {
                    $\varPsi_{0,i} = \varPsi\left(x_{0,i}\text{, }y_{0,i}\right)$
                };
                \node[draw, rounded corners, minimum width=1cm, minimum height=8mm, right=of D] (E) {
                    $\varPsi_{\text{S},i}\left(x,y\right) = \varPsi_{0,i}$
                };
                \draw[->] (A) -- (B);
                \draw[->] (B) -- (C);
                \draw[->] (C) -- (D);
                \draw[->] (D) -- (E);
            \end{tikzpicture}
        }{0}{0}{0}{0}
\end{tikzpicture}

% ### Zweite Seite der Potentialtheorie ###
\newpage
\begin{center}
    \section*{\underline{Potentialtheorie}}
    \titlebf{Singularitätentabelle}
\end{center}

\vspace*{-4mm}

\begin{table}[H]
    \centering
    \newlength{\rowhght}
    \setlength{\rowhght}{6.5mm}
    \footnotesize
    \renewcommand{\arraystretch}{2.0}
    \resizebox{\textwidth-20mm}{!}{\begin{tabularx}{205mm}{|l|c|c|c|c|Z|Z|Z|c|}
        \cline{1-9}
        \multirow{4}{*}{Bezeichnung} & \multirow{4}{*}{Stromlinienbild} & \multirow{2}{*}{\begin{tabular}[c]{@{}c@{}}\scriptsize Komplexe\\[-2.6ex]\scriptsize Potentialfunktion\end{tabular}} & \multicolumn{2}{c|}{Skalare} & \multicolumn{4}{c|}{\multirow{2}{*}{Geschwindigkeitskomponenten}} \\ 
        \cline{4-5}
         &  &  & \scriptsize Potentialfunktion & \scriptsize Stromfunktion & \multicolumn{4}{c|}{} \\ 
        \cline{3-9}
         &  & $F(z)$ & $\varPhi(x,y)$ & $\varPsi(x,y)$ & $u(x,y)$ & $v(x,y)$ & $v_r(x,y)$ & $v_\varphi(x,y)$ \\ 
        \cline{3-9}
         &  & $F(r,\varphi)$ & $\varPhi(r,\varphi)$ & $\varPsi(r,\varphi)$ & $u(r,\varphi)$ & $v(r,\varphi)$ & $v_r(r,\varphi)$ & $v_\varphi(r,\varphi)$ \\ 
        \Cline{1-9}{1.2pt}
        \xrowht[()]{\rowhght}\multirow{3}{*}{\begin{tabular}[l]{@{}l@{}}\\[\dimexpr(\rowhght/2)-10.8ex]Translations-\\[-2.6ex]Strömung in\\[-2.6ex]$x$-Richtung\end{tabular}} & \multirow{2}{*}{} & $u_\infty\,z$ & $u_\infty\,x$ & $u_\infty\,y$ & $u_\infty$ & $0$ & $u_\infty\,\tfrac{x}{\sqrt{x^2+y^2}}$ & $-u_\infty\,\tfrac{y}{\sqrt{x^2+y^2}}$ \\
        \cline{3-9}
        \xrowht[()]{\rowhght} &  & $u_\infty\,r\,e^{i\varphi}$ & $u_\infty\,r\,\cos(\varphi)$ & $u_\infty\,r\,\sin(\varphi)$ & $u_\infty$ & $0$ & $u_\infty\,\cos(\varphi)$ & $-u_\infty\,\sin(\varphi)$ \\ 
        \Cline{1-9}{1.2pt}
        \xrowht[()]{\rowhght}\multirow{3}{*}{\begin{tabular}[l]{@{}l@{}}\\[\dimexpr(\rowhght/2)-10.8ex]Translations-\\[-2.6ex]Strömung in\\[-2.6ex]$y$-Richtung\end{tabular}} & \multirow{2}{*}{} & $-i\,v_\infty\,z$ & $v_\infty\,y$ & $-v_\infty\,x$ & $0$ & $v_\infty$ & $v_\infty\,\tfrac{y}{\sqrt{x^2+y^2}}$ & $v_\infty\,\tfrac{x}{\sqrt{x^2+y^2}}$ \\ 
        \cline{3-9}
        \xrowht[()]{\rowhght} &  & $-i\,v_\infty\,r\,e^{i\varphi}$ & $v_\infty\,r\,\sin(\varphi)$ & $-v_\infty\,r\,\cos(\varphi)$ & $0$ & $v_\infty$ & $v_\infty\,\sin(\varphi)$ & $v_\infty\,\cos(\varphi)$ \\ 
        \Cline{1-9}{1.2pt}
        \xrowht[()]{\rowhght}\multirow{3}{*}{\begin{tabular}[l]{@{}l@{}}\\[\dimexpr(\rowhght/2)-10.8ex]Staupunkt-\\[-2.6ex]\& Eckenstr.\\[-2.6ex]$a\in\mathbb{R}^+$\end{tabular}} & \multirow{2}{*}{} & $\tfrac{a}{2}\,z^2$ & $\tfrac{a}{2}\left(x^2-y^2\right)$ & $a\,x\,y$ & $a\,x$ & $-a\,y$ & $a\,\tfrac{x^2-y^2}{\sqrt{x^2+y^2}}$ & $-a\,\tfrac{2\,x\,y}{\sqrt{x^2+y^2}}$ \\
        \cline{3-9}
        \xrowht[()]{\rowhght} &  & $\tfrac{a}{2}\,r^2\,e^{2i\varphi}$ & $\tfrac{a}{2}\,r^2\,\cos(2\,\varphi)$ & $\tfrac{a}{2}\,r^2\,\sin(2\,\varphi)$ & $a\,r\,\cos(\varphi)$ & $-a\,r\,\sin(\varphi)$ & $a\,r\,\cos(2\,\varphi)$ & $-a\,r\,\sin(2\,\varphi)$ \\
        \Cline{1-9}{1.2pt}
        \xrowht[()]{\rowhght}\multirow{3}{*}{\begin{tabular}[l]{@{}l@{}}\\[\dimexpr(\rowhght/2)-10.8ex]Quelle, Senke\\[-2.6ex](Ergiebigkeit\\[-2.6ex]sei $Q \neq 0$)\end{tabular}} & \multirow{2}{*}{} & $a$ & $a$ & $a$ & $a$ & $a$ & $a$ & $a$ \\
        \cline{3-9}
        \xrowht[()]{\rowhght} &   & $a$ & $a$ & $a$ & $a$ & $a$ & $a$ & $a$ \\
        \Cline{1-9}{1.2pt}
        \xrowht[()]{\rowhght}\multirow{3}{*}{\begin{tabular}[l]{@{}l@{}}\\[\dimexpr(\rowhght/2)-10.8ex]\scriptsize Potentialwirbel\\[-2.6ex](Zirkulation\\[-2.6ex]sei $\Gamma\neq 0$)\end{tabular}} & \multirow{2}{*}{} & $a$ & $a$ & $a$ & $a$ & $a$ & $a$ & $a$ \\
        \cline{3-9}
        \xrowht[()]{\rowhght} &   & $a$ & $a$ & $a$ & $a$ & $a$ & $a$ & $a$ \\
        \Cline{1-9}{1.2pt}
        \xrowht[()]{\rowhght}\multirow{3}{*}{\begin{tabular}[l]{@{}l@{}}\\[\dimexpr(\rowhght/2)-10.8ex]Dipol\\[-2.6ex](Auf der\\[-2.6ex]$x$-Achse)\end{tabular}} & \multirow{2}{*}{} & $a$ & $a$ & $a$ & $a$ & $a$ & $a$ & $a$ \\
        \cline{3-9}
        \xrowht[()]{\rowhght} &   & $a$ & $a$ & $a$ & $a$ & $a$ & $a$ & $a$ \\
        \Cline{1-9}{1.2pt}
        \xrowht[()]{\rowhght}\multirow{3}{*}{\begin{tabular}[l]{@{}l@{}}\\[\dimexpr(\rowhght/2)-10.8ex]Dipol\\[-2.6ex](Auf der\\[-2.6ex]$y$-Achse)\end{tabular}} & \multirow{2}{*}{} & $a$ & $a$ & $a$ & $a$ & $a$ & $a$ & $a$ \\
        \cline{3-9}
        \xrowht[()]{\rowhght} &   & $a$ & $a$ & $a$ & $a$ & $a$ & $a$ & $a$ \\
        \cline{1-9}
    \end{tabularx}}
\end{table}
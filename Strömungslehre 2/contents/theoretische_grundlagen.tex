% Diese Seite verwendet eine eigene Seitengeometrie und verzichtet auf die Fußnote.
\newgeometry{left=2cm, right=2cm, top=1cm, bottom=1cm}
\thispagestyle{empty}

% Überschrift der Seite, zentriert
\begin{center}
    \section*{\underline{Theoretische Grundlagen}}
\end{center}

\begin{raggedright}

\subsection*{\myul{Terminologie}} \vspace*{-2mm}
\begin{table}[H]
    \begin{tabular}{l l l}
        \textbf{Begriff} & \textbf{Bedeutung} & \textbf{Mathematisch} \\[2.0ex]
        Stromlinie   & \begin{tabular}[c]{@{}l@{}}Linie, die zu jedem Zeitpunkt tangential zur Geschwindigkeit\\ verläuft (siehe Strömung I)\end{tabular} & $\Psi =$ konstant \\[3.0ex]
        Wirbellinie  & \begin{tabular}[c]{@{}l@{}}Laufen zu einer bestimmten Zeit an jedem Ort tangential zum\\ Wirbel- oder Drehvektor $\underline{\omega}$ (analog zur Stromlinie).\end{tabular} & - \\[2.0ex]
        Wirbelfaden  & \begin{tabular}[c]{@{}l@{}}Gesamtheit der Wirbellinien, die durch eine Fläche $A_1$ ein-\\ und eine Fläche $A_2$ austreten.\end{tabular} & - \\[2.0ex]
        Wirbelröhre  & \begin{tabular}[c]{@{}l@{}}Wirbellinien durch die Punkte einer geschlossenen Kurve bilden\\ die Mantelfläche der Wirbelröhre.\end{tabular} & - \\[2.0ex]
        Wirbelstrom  & Analog zum Volumenstrom. & $\underline{\omega}\cdot\underline{A} =$ konstant \\[3.0ex]
        Isobare      & Linie gleicher statischer Drücke in der Strömung. & $p =$ konstant \\[3.0ex]
        Isotache     & Linie gleicher Geschwindigkeit in der Strömung. & $\left|\underline{v}\right| =$ konstant \\[3.0ex]
        Isokline     & Linie gleicher Stromlinienneigung in der Strömung. & $\tan(\alpha) = \dfrac{v}{u} =$ konstant \\
    \end{tabular}
\end{table}

\end{raggedright}

\restoregeometry
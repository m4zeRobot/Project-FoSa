% ### Erste Seite der Gasdynamik ###
\begin{center}
    \section*{\underline{Gasdynamik}}
\end{center}

\begin{tikzpicture}[remember picture, overlay]

% --- ERSTE REIHE: Am Seitenursprung anfangen
\setlength{\rowheight}{17mm}

    % A1) Erste Kachel relativ zum Seitenursprung
    \setlength{\cellwidth}{52mm}
    \Tile{A1}{([xshift=10mm, yshift=-20mm]current page.north west)}{\cellwidth}{\rowheight}
        {
            \vspace{-2mm}
            \titlebf{Massenerhaltung}\\ \vspace{-3mm}
            \begin{equation}
                \frac{\partial(\rho\,A)}{\partial t}+\frac{\partial(\rho\,A\,V)}{\partial x}=0 \tag*{\color{NavyBlue}(9.4b)}
            \end{equation}
        }{0}{1}{1}{0}

    % B1) Zweite Kachel relativ zu A1
    \setlength{\cellwidth}{42mm}
    \Tile{B1}{(A1.north east)}{\cellwidth}{\rowheight}
        {
            \vspace{-2mm}
            \titlebf{Impulssatz}\\ \vspace{-3mm}
            \begin{equation}
                \rho\,\frac{\text{D}V}{\text{D}t} = -\frac{\partial p}{\partial x} \tag*{\color{NavyBlue}(9.5)}
            \end{equation}
        }{0}{1}{1}{0}

    % C1) Dritte Kachel relativ zu B1
    \setlength{\cellwidth}{48mm}
    \Tile{C1}{(B1.north east)}{\cellwidth}{\rowheight}
        {
            \vspace{-2mm}
            \titlebf{Energieerhaltung}\\ \vspace{-3mm}
            \begin{equation*}
                \frac{\text{D}h_0}{\text{D}t} = \frac{1}{\rho}\,\frac{\partial p}{\partial t}
            \end{equation*}
        }{0}{1}{1}{0}

    % D1) Dritte Kachel relativ zu C1
    \setlength{\cellwidth}{48mm}
    \Tile{D1}{(C1.north east)}{\cellwidth}{\rowheight}
        {
            \vspace{-2mm}
            \titlebf{Entropieänderung}\\ \vspace{-3mm}
            \begin{equation*}
                \frac{\text{D}s}{\text{D}t} = 0
            \end{equation*}
        }{0}{0}{1}{0}

% --- ZWEITE REIHE: relativ zu A1
\setlength{\rowheight}{22mm}

    % A2) Erste Kachel relativ zu A1
    \setlength{\cellwidth}{64mm}
    \Tile{A2}{(A1.south west)}{\cellwidth}{\rowheight}
        {
            \titlebf{Schallgeschwindigkeit}\\ \vspace{-2mm}
            \begin{equation*}
                c = \sqrt{\kappa\,R\,T} = \sqrt{\kappa\,\frac{p}{\rho}}
            \end{equation*}
        }{0}{1}{1}{0}

    % B2) Zweite Kachel relativ zu A2
    \setlength{\cellwidth}{126mm}
    \Tile{B2}{(A2.north east)}{\cellwidth}{\rowheight}
        {
            \titlebf{Stationäre Stromröhre}\\ \vspace{1mm}
            \begin{tabularx}{\dimexpr\cellwidth-2\TileSep}{Z|Z|Z}
                \titlerm{Massenerhaltung} & \titlerm{Impulssatz} & \titlerm{Energiegleichung} \\[1.0ex]
                $\dot m = \rho\,A\,V = \text{konst.}$ & $\displaystyle\frac{V^2}{2}+\int\,\frac{1}{\rho}\,\text{d}p = \text{konst.}$ & $h_0 = h+\dfrac{V^2}{2} = \text{konst.}$ \\[-1.5ex]
                 &  & 
            \end{tabularx}
        }{0}{0}{1}{0}

% --- DRITTE REIHE: relativ zu A2
\setlength{\rowheight}{110mm}

    % A3) Erste Kachel relativ zu A2
    \setlength{\cellwidth}{62mm}
    \setlength{\TileSep}{0mm}
    \Tile{A3}{(A2.south west)}{\cellwidth}{\rowheight}
        {
            \vspace{2mm}
            \titlerm{Kritische Schallgeschwindigkeit}\\ \vspace{-4mm}
            \begin{equation*}
                c_\text{L} = c^* = \sqrt{\frac{2}{\kappa+1}}\,c_0 = \sqrt{2\,\frac{\kappa}{\kappa+1}\,R\,T_0}
            \end{equation*}
            \vspace{-5.5mm}\\
            \raggedright
            \hspace{3mm}{\color{NavyBlue}(9.22a) \rotatebox[origin=c]{270}{$\Lsh$}}\\ \vspace{-4mm}
            \centering
            \begin{equation*}
                \left(\frac{V}{c^*}\right)^2=\frac{\kappa+1}{\kappa-1}\left(1-\frac{T}{T_0}\right)=\frac{\kappa+1}{\frac{2}{\text{Ma}^2}+\kappa-1}
            \end{equation*}
            \vspace{-3.5mm}\\
            {\color{NavyBlue}(9.21b) \rotatebox[origin=c]{270}{$\Lsh$}}\\[0.5ex]
            \includegraphics[width=55mm]{adiabate-stroemung.pdf}
        }{0}{1}{1}{0}
    \setlength{\TileSep}{2mm}

    % B3) Zweite Kachel relativ zu A3
    \setlength{\cellwidth}{128mm}
    \setlength{\TileSep}{0mm}
    \Tile{B3}{(A3.north east)}{\cellwidth}{\rowheight}
        {
            \vspace{2mm}
            {\color{RedOrange}astrr()}\hspace{24mm}\titlebf{Adiabate Strömungsvorgänge}\hspace{24mm}{\color{RedOrange}astrkr()}\\ \vspace{2mm}
            \begin{tabularx}{\dimexpr\cellwidth-2\TileSep}{c|Z}
                \titlerm{Ruhezustand} & \titlerm{Krit. Zustand (Ma = 1)} \\[1.0ex]
                \qquad$\dfrac{T_0}{T} = 1 + \dfrac{\kappa-1}{2}\,\text{Ma}^2$~~{\color{NavyBlue}(9.19)} & $\dfrac{T^*}{T_0}=\dfrac{2}{\kappa+1}$ \\[2.5ex]
                $\dfrac{p_0}{p}=\left(\dfrac{T_0}{T}\right)^\frac{\kappa}{\kappa-1}=\left(1+\dfrac{\kappa-1}{2}\,\text{Ma}^2\right)^\frac{\kappa}{\kappa-1}$ & \vspace{-2mm}$\dfrac{p^*}{p_0}=\left(\dfrac{2}{\kappa+1}\right)^\frac{\kappa}{\kappa-1}$ \\[-1.0ex]
                {\color{NavyBlue}(9.20a) \rotatebox[origin=c]{270}{$\Lsh$}} &  \\[2.0ex]
                $\dfrac{\rho_0}{\rho}=\left(\dfrac{T_0}{T}\right)^\frac{1}{\kappa-1}=\left(1+\dfrac{\kappa-1}{2}\,\text{Ma}^2\right)^\frac{1}{\kappa-1}$ & $\dfrac{\rho^*}{\rho_0}=\left(\dfrac{2}{\kappa+1}\right)^\frac{1}{\kappa-1}$ \\[2.0ex]
                {\color{NavyBlue}(9.20b) \rotatebox[origin=c]{270}{$\Lsh$}} & \vspace{1mm}{\color{NavyBlue}\rotatebox[origin=c]{90}{$\Lsh$} (9.26)} \\[2.0ex]
                $h_0 = h+\dfrac{V^2}{2} \Leftrightarrow T_0 = T+\dfrac{V^2}{2\,c_\text{p}}$~~{\color{NavyBlue}(8.60)} & $\dfrac{A}{A^*}=\dfrac{1}{\text{Ma}}\left[\dfrac{2}{\kappa+1}\left(1+\dfrac{\kappa-1}{2}\text{Ma}^2\right)\right]^\frac{\kappa+1}{2\,(\kappa-1)}$ \\[3.0ex]%
            \end{tabularx}
            \vspace{-2.5mm}\\
            \setlength{\TileSep}{2mm}
            \rule{\dimexpr\cellwidth-2\TileSep}{0.5pt}\\
            \raggedright
            {\hspace{3mm}\color{RedOrange}mstr()}\\ \vspace{-6mm}
            \centering
            \begin{equation*}
                \dot m=\frac{\text{Ma}}{\left(1+\frac{\kappa-1}{2}\,\text{Ma}^2\right)^\frac{\kappa+1}{2\,(\kappa-1)}}\,\sqrt{\frac{\kappa}{R\,T_0}}\,p_0\,A = \left(\frac{2}{\kappa+1}\right)^{\frac{\kappa+1}{2\,(\kappa-1)}}\,\sqrt{\frac{\kappa}{R\,T_0}}\,p_0\,A^*\hspace{3mm}\text{\color{NavyBlue}(9.24)}
            \end{equation*}
            \vspace{-4mm}\\
            \rule{\dimexpr\cellwidth-2\TileSep}{0.5pt} \\ \vspace{1.5mm}
            \begin{tabularx}{\dimexpr\cellwidth-2\TileSep}{Z|c}
                \titlebf{Machzahl} & \vspace{-2.5mm}\hspace{6mm}{\color{RedOrange}la()}\hspace{6mm}\titlebf{Lavalzahl}\hspace{12mm}{\color{NavyBlue}(9.23b) \rotatebox[origin=c]{270}{$\Rsh$}} \\[3.0ex]
                $\text{Ma} = \dfrac{V}{c}$ & \begin{tabular}{c|c}
                    $\text{La} = \dfrac{V}{c^*} = \sqrt{\dfrac{\kappa+1}{2+(\kappa-1)\,\text{Ma}^2}}\,\text{Ma}$ & $\text{La}_\text{max} = \sqrt{\dfrac{\kappa+1}{\kappa-1}}$ \\[2.5ex]
                    {\color{NavyBlue}(9.23a) \rotatebox[origin=c]{270}{$\Lsh$}} & $\text{Luft: La}_\text{max} = 2,449$ \\[1.5ex]%
                 \end{tabular}
            \end{tabularx}
        }{0}{0}{1}{0}

\setlength{\TileSep}{2mm}
    
% --- VIERTE REIHE: relativ zu A3
\setlength{\rowheight}{100mm}

    % X4) EIGENES FELD für die Zeilen-Überschrift
    \Tile{X4}{(A3.south west)}{190mm}{8mm}
        {
            \vspace{2mm}
            \titlebf{Stationäre senkrechte Verdichtungsstöße}
        }{0}{0}{0}{0}

    % A4) Erste Kachel relativ zu X4
    \setlength{\cellwidth}{49mm}
    \Tile{A4}{(X4.south west)}{\cellwidth}{\rowheight}
        {
            \titlerm{Erhaltungssätze}
            \begin{align*}
                \rho_1\,V_1 &= \rho_2\,V_2 \\[1.0ex]
                \rho_1\,V_1^2+p_1 &= \rho_2\,V_2^2+p_2 \\[1.0ex]
                h_1+\frac{V_1^2}{2} &= h_2+\frac{V_2^2}{2}
            \end{align*}
            \vspace{-1mm}\\
            \includegraphics[width=45mm]{erhaltungssaetze.pdf}
            \rule{\dimexpr\cellwidth-2\TileSep}{0.5pt} \\ \vspace{1.5mm}
            \titlerm{Entropiezunahme}
            \begin{align*}
                s_2-s_1 &= R\,\ln\left(\frac{\rho_1}{\rho_2}\left(\frac{T_2}{T_1}\right)^\frac{1}{\kappa-1}\right) \\[0.5ex]
                &= R\,\ln\left(\frac{p_{01}}{p_{02}}\right)
            \end{align*}
            {\color{NavyBlue}(9.40) \rotatebox[origin=c]{270}{$\Lsh$}}
        }{0}{1}{0}{0}

    % B4) Zweite Kachel relativ zu A4
    \setlength{\cellwidth}{90.5mm}
    \Tile{B4}{(A4.north east)}{\cellwidth}{\rowheight}
        {
            \hspace{20mm} \titlerm{Stoßbeziehungen} \hspace{12mm} {\color{RedOrange}ssv()}
            \begin{align*}
                \hspace{6mm}\text{\color{RedOrange}\raisebox{.5pt}{\textcircled{\raisebox{-0.3pt}{\scriptsize 1}}}}\hspace{7mm}\frac{p_2}{p_1} &= \frac{2\,\kappa\,\text{Ma}_1^2-\kappa+1}{\kappa+1}\hspace{6mm}\text{\color{NavyBlue}(9.47a)} \\[0.5ex]
                \hspace{6mm}\text{\color{RedOrange}\raisebox{.5pt}{\textcircled{\raisebox{-0.3pt}{\scriptsize 2}}}}\hspace{7mm}\frac{\rho_2}{\rho_1} &= \frac{(\kappa+1)\,\text{Ma}_1^2}{2+(\kappa-1)\,\text{Ma}_1^2}\hspace{6mm}\text{\color{NavyBlue}(9.38b)}
            \end{align*}
            \vspace{-5mm}\\
            {\color{NavyBlue}(9.38a) \rotatebox[origin=c]{270}{$\Rsh$}} \hspace{66mm}
            \vspace{-2mm}
            \begin{equation*}
                \text{\color{RedOrange}\raisebox{.5pt}{\textcircled{\raisebox{-0.3pt}{\scriptsize 3}}}}\hspace{3mm} \frac{T_2}{T_1}=\frac{h_2}{h_1} = \frac{V_2^2}{V_1^2} = \frac{(2\,\kappa\,\text{Ma}_1^2-\kappa+1)(2+(\kappa-1)\,\text{Ma}_1^2)}{(\kappa+1)^2\,\text{Ma}_1^2}
            \end{equation*}
            \begin{equation*}
                \hspace{6mm}\text{\color{RedOrange}\raisebox{.5pt}{\textcircled{\raisebox{-0.3pt}{\scriptsize 4}}}}\hspace{5mm}\text{Ma}_2=\sqrt{\frac{(\kappa-1)(\text{Ma}_1^2-1)+\kappa+1}{2\,\kappa\,(\text{Ma}_1^2-1)+\kappa+1}}\hspace{4mm}\text{\color{NavyBlue}(9.35)}
            \end{equation*}
            \begin{equation*}
                \hspace{4mm}\text{\color{RedOrange}\raisebox{.5pt}{\textcircled{\raisebox{-0.3pt}{\scriptsize 5}}}}\hspace{3mm} \frac{p_{02}}{p_{01}} = \left[\frac{\left(\frac{\kappa+1}{2}\right)^{\kappa+1}\,\text{Ma}_1^{(2\,\kappa)}}{(1+\frac{\kappa-1}{2}\,\text{Ma}_1^2)^\kappa\,(\kappa\,\text{Ma}_1^2-\frac{\kappa-1}{2})}\right]^{\frac{1}{\kappa-1}}\hspace{2mm}\text{\color{NavyBlue}(9.39b)}
            \end{equation*}
            \vspace{-1mm}\\
            \rule{\dimexpr\cellwidth-2\TileSep}{0.5pt} \\ \vspace{1.5mm}
            \hspace{16mm}\titlerm{Rayleigh-Pitot-Formel}\hspace{10mm}{\color{RedOrange}rpf()}\\ \vspace{-1mm}
            \begin{equation*}
                \hspace{14mm}\frac{p_\text{pitot}}{p_1} = \left[\frac{\left(\frac{\kappa+1}{2}\right)^{\kappa+1}\,\text{Ma}_1^{(2\,\kappa)}}{\kappa\,\text{Ma}_1^2-\frac{\kappa-1}{2}}\right]^{\frac{1}{\kappa-1}}\hspace{6mm}{\color{NavyBlue}(9.41)}
            \end{equation*}
        }{0}{1}{0}{0}

    % C4) Zweite Kachel relativ zu B4
    \setlength{\cellwidth}{50.5mm}
    \Tile{C4}{(B4.north east)}{\cellwidth}{\rowheight}
        {
            \titlerm{Rankine-Hugoniot-Beziehung}
            \begin{equation*}
                e_2-e_1 = \frac{p_1+p_2}{2}\,(v_1-v_2)
            \end{equation*}
            \vspace{-5mm}\\
            {\color{NavyBlue}\rotatebox[origin=c]{90}{$\Rsh$} (9.53) \hspace{22mm}}\\ \vspace{-2mm}
            {\color{NavyBlue}\hspace{22mm} (9.54) \rotatebox[origin=c]{270}{$\Rsh$}}\\ \vspace{-5mm}
            \begin{equation*}
                \frac{p_2}{p_1} = \frac{(\kappa+1)\,\rho_2-(\kappa-1)\,\rho_1}{(\kappa+1)\,\rho_1-(\kappa-1)\,\rho_2}
            \end{equation*}
            \rule{\dimexpr\cellwidth-2\TileSep}{0.5pt} \\ \vspace{1.5mm}
            \titlerm{Rayleigh-Gerade}
            \begin{equation*}
                \frac{p_2-p_1}{v_2-v_1} = -\left(\frac{u_1}{v_1}\right)^2
            \end{equation*}
            \vspace{1mm}\\
            \includegraphics[width=46mm]{rayleigh-gerade.pdf}
        }{0}{0}{0}{0}

\end{tikzpicture}

\newpage
% ### Zweite Seite der Gasdynamik ###
\begin{center}
    \section*{\underline{Gasdynamik}}
\end{center}

\begin{tikzpicture}[remember picture, overlay]

% --- ERSTE REIHE: Am Seitenursprung anfangen
\setlength{\rowheight}{95mm}

    % X1) EIGENES FELD für die Zeilen-Überschrift
    \Tile{X1}{([xshift=10mm, yshift=-14mm]current page.north west)}{190mm}{9mm}
        {
            \vspace{2mm}
            \titlebf{Schräge Verdichtungsstöße}%
        }{0}{0}{0}{0}

    % A1) Erste Kachel relativ zu X1
    \setlength{\cellwidth}{108.5mm}
    \Tile{A1}{(X1.south west)}{\cellwidth}{\rowheight}
        {
            \vspace{-1mm}
            \hspace{14mm}\titlerm{Stoßbeziehungen}\hspace{9mm}{\color{RedOrange}sv()}\\ \vspace{1mm}
            \begin{minipage}{0.42\cellwidth-\TileSep}
                \centering
                \vspace{1mm}%
                \includegraphics[width=\textwidth-2\TileSep]{senkrechter-stoss.pdf}\\
                \vspace{1mm}
                {\color{NavyBlue}(9.43) \rotatebox[origin=c]{270}{$\Rsh$}}%
                \vspace{-2mm}
                \begin{align*}
                    \text{Ma}_{1n} &= \text{Ma}_1\,\sin(\sigma) \\
                    \text{Ma}_{2n} &= \text{Ma}_2\,\sin(\sigma-\vartheta)
                \end{align*}
            \end{minipage}%
            \hspace{\TileSep}\vrule\hspace{\TileSep}
            \begin{minipage}{0.52\cellwidth-\TileSep}
                \centering
                {\color{NavyBlue}(9.47a) \rotatebox[origin=c]{270}{$\Rsh$}}%
                \vspace{-1.2mm}
                \begin{equation*}
                    \frac{p_2}{p_1}=1+\frac{2\,\kappa}{\kappa+1}\,\left(\text{Ma}_1^2\,\sin^2(\sigma)-1\right)
                \end{equation*}
                \begin{equation*}
                    \frac{\rho_2}{\rho_1} = \frac{u_1}{u_2} = \frac{(\kappa+1)\,\text{Ma}_1^2\,\sin^2(\sigma)}{2+(\kappa-1)\,\text{Ma}_1^2\,\sin^2(\sigma)}
                \end{equation*}
                \\ \vspace{1mm}
                {\color{NavyBlue}(9.44) \rotatebox[origin=c]{270}{$\Lsh$}}
            \end{minipage}
            \\ \vspace{1mm}
            \rule{\dimexpr\cellwidth-2\TileSep}{0.5pt}
            \begin{equation*}
                \frac{T_2}{T_1} = \frac{V_2^2}{V_1^2} = \frac{p_2}{p_1}\,\frac{\rho_1}{\rho_2} = \frac{(2\,\kappa\,\text{Ma}_1^2\,\sin^2(\sigma)-\kappa+1)\,(2+(\kappa-1)\,\text{Ma}_1^2\,\sin^2(\sigma))}{(\kappa+1)^2\,\text{Ma}_1^2\,\sin^2(\sigma)}
            \end{equation*}%
            \vspace{1mm}
            \begin{equation*}
                \frac{\text{Ma}_1}{\text{Ma}_2} = \left[\frac{1+\frac{2\,\kappa}{\kappa+1}\,(\text{Ma}_1^2\,\sin^2(\sigma)-1)}{1-\frac{2}{\kappa+1}\,\left(1-\frac{1}{\text{Ma}_1^2\,\sin^2(\sigma)}\right)}\right]^{\frac{1}{2}}\frac{\sin(\sigma-\vartheta)}{\sin(\sigma)}\quad\text{\color{NavyBlue}(9.59)}
            \end{equation*}%
            \vspace{2mm}
            \begin{equation*}
                \frac{p_{02}}{p_{01}} = \left[\frac{\left(\frac{\kappa+1}{2}\right)^{\kappa+1}\,\left(\text{Ma}_1\,\sin(\sigma)\right)^{2\kappa}}{\left(1+\frac{\kappa-1}{2}\,(\text{Ma}_1\,\sin(\sigma))^2\right)^\kappa\,\left(\kappa\,(\text{Ma}_1\,\sin(\sigma))^2-\frac{\kappa-1}{2}\right)}\right]^\frac{1}{\kappa-1}~~\text{\color{NavyBlue}(9.58)}
            \end{equation*}
        }{0}{1}{1}{0}

    % B1) Zweite Kachel relativ zu A1
    \setlength{\cellwidth}{81.5mm}
    \Tile{B1}{(A1.north east)}{\cellwidth}{\rowheight}
        {
            \vspace{-1mm}
            \titlerm{Mach'sche Welle}\\ \vspace{2mm}
            \begin{minipage}{0.41\cellwidth-\TileSep}
                \centering
                \vspace{-1mm}
                {\color{NavyBlue}(9.67) \rotatebox[origin=c]{270}{$\Rsh$}}\hspace{18mm}~\\ \vspace{-6mm}
                \begin{equation*}
                    \mu = \arcsin\left(\frac{1}{\text{Ma}}\right)
                \end{equation*}
                \\ \vspace{1mm}
                Es gilt: $\mu\leq\sigma\leq\frac{\pi}{2}$\\ \vspace{2mm}
            \end{minipage}%
            \hspace{0.5\TileSep}\vrule\hspace{0.5\TileSep}
            \begin{minipage}{0.51\cellwidth-\TileSep}
                \centering
                \includegraphics[width=39.5mm]{schraeger-stoss.pdf}
            \end{minipage}
            \\ \vspace{1mm}
            \rule{\dimexpr\cellwidth-2\TileSep}{0.5pt}\\ \vspace{1mm}
            \titlerm{Umlenkwinkel}
            \begin{equation*}
                \tan(\vartheta) = \cot(\sigma)\,\frac{\text{Ma}_1^2\,\sin^2(\sigma)-1}{1+(\frac{\kappa+1}{2}-\sin^2(\sigma))\,\text{Ma}_1^2}~~\text{\color{NavyBlue}(9.49)}
            \end{equation*}
            \rule{\dimexpr\cellwidth-2\TileSep}{0.5pt}\\ \vspace{1mm}
            \titlerm{Entropiezunahme (Ideales Gas)}
            \begin{equation*}
                s_2-s_1 = c_\text{v}\,\ln\left(\frac{p_2}{p_1}\left(\frac{\rho_1}{\rho_2}\right)^\kappa\right)~~\text{\color{NavyBlue}(9.47c)}
            \end{equation*}
            \rule{\dimexpr\cellwidth-2\TileSep}{0.5pt}\\
            \hspace{12mm}\titlebf{Schräge Stoßexpansion}\hspace{6mm}{\color{RedOrange}schs()}\\ \vspace{2mm}
            \begin{minipage}{0.40\cellwidth-\TileSep}
                \centering
                \vspace{-1mm}
                \begin{equation*}
                    \frac{T_2}{T_1} = \frac{1+\frac{\kappa-1}{2}\,\text{Ma}_1^2}{1+\frac{\kappa-1}{2}\,\text{Ma}_2^2}
                \end{equation*}
                \\ \vspace{-1.5mm}
                {\color{NavyBlue}(9.77a) \rotatebox[origin=c]{270}{$\Lsh$}}
            \end{minipage}%
            \hspace{0.5\TileSep}\vrule\hspace{0.5\TileSep}
            \begin{minipage}{0.54\cellwidth-\TileSep}
                \centering
                \vspace{-1mm}
                \begin{equation*}
                    \frac{p2}{p_1} = \left(\frac{1+\frac{\kappa-1}{2}\,\text{Ma}_1^2}{1+\frac{\kappa-1}{2}\,\text{Ma}_2^2}\right)^\frac{\kappa}{\kappa-1}
                \end{equation*}
                \\ \vspace{-1.5mm}
                {\color{NavyBlue}(9.77b) \rotatebox[origin=c]{270}{$\Lsh$}}
            \end{minipage}
        }{0}{0}{1}{0}

% --- ZWEITE REIHE: relativ zu A1
\setlength{\rowheight}{23mm}

    % A2) Erste Kachel relativ zu A1
    \setlength{\cellwidth}{40mm}
    \Tile{A2}{(A1.south west)}{\cellwidth}{\rowheight}
        {
            \titlerm{Öffnungskegel}\\ \vspace{-2.5mm}
            \begin{equation*}
                \delta = \mu_2 - \mu_1 + \Delta\vartheta_{12}
            \end{equation*}
        }{0}{1}{1}{0}

    % B2) Zweite Kachel relativ zu A2
    \setlength{\cellwidth}{150mm}
    \Tile{B2}{(A2.north east)}{\cellwidth}{\rowheight}
        {
            \hspace{14mm}\titlerm{Prandtl-Meyer-Funktion}\hspace{7mm}{\color{RedOrange}pmf()}\\ \vspace{1mm}
            \begin{tabularx}{\dimexpr\cellwidth-2\TileSep}{c|Z}
                 & \\[-2.5ex]
                $\nu(\text{Ma}_2) = \nu(\text{Ma}_1) + \vartheta_{12}$ & $\displaystyle\nu(\text{Ma}) = \sqrt{\frac{\kappa+1}{\kappa-1}}\,\arctan\left(\sqrt{\frac{\kappa-1}{\kappa+1}\,(\text{Ma}^2-1)}\right)-\arctan\left(\sqrt{\text{Ma}^2-1}\right)$ \\[-2.0ex]
                {\color{NavyBlue}(9.75) \rotatebox[origin=c]{270}{$\Lsh$}} & {\color{NavyBlue}(9.74) \rotatebox[origin=c]{270}{$\Lsh$}}
            \end{tabularx}
        }{0}{0}{1}{0}

% --- DRITTE REIHE: relativ zu A2
\setlength{\rowheight}{135mm}

    % A3) Erste Kachel relativ zu A2
    \setlength{\cellwidth}{110mm}
    \Tile{A3}{(A2.south west)}{\cellwidth}{\rowheight}
        {
            \titlebf{Instationäre eindimensionale Strömung}\\ \vspace{1mm}
            \begin{tabularx}{\dimexpr\cellwidth-2\TileSep}{Z|c}
                \titlerm{Kolbengeschwindigkeit} & \titlerm{Machzahl} \\[2.0ex]
                $\dfrac{V_\text{k}'}{c_1} = \dfrac{2}{\kappa+1}\,\dfrac{\text{Ma}_\text{s}^2-1}{\text{Ma}_\text{s}}$ & $\text{Ma}_{2}' = \dfrac{\text{Ma}_\text{s}^2-1}{\sqrt{(1+\frac{\kappa-1}{2}\,\text{Ma}_\text{s}^2)\,(\kappa\,\text{Ma}_\text{s}^2-\frac{\kappa-1}{2})}}$ \\[-1.5ex]
                 & 
            \end{tabularx}
            \\ \vspace{1mm}
            \rule{\dimexpr\cellwidth-2\TileSep}{0.5pt}\\
            \titlerm{Stoßbeziehungen}
            \begin{equation*}
                \frac{p_2'}{p_1'} = \frac{2\,\kappa\,\text{Ma}_\text{s}^2-\kappa+1}{\kappa+1}
            \end{equation*}
            \begin{equation*}
                \frac{p_{02}'}{p_1'} = \left(\frac{\kappa+1}{2\,\kappa\,\text{Ma}_\text{s}^2-\kappa+1}\right)^\frac{1}{\kappa-1}\,\left(\text{Ma}_\text{s}^2\,\frac{2\,(\kappa-1)\,\text{Ma}_\text{s}^2+(3-\kappa)}{2+(\kappa-1)\,\text{Ma}_\text{s}^2}\right)^\frac{\kappa}{\kappa-1}
            \end{equation*}%
            \vspace{2mm}
            \begin{equation*}
                \frac{T_2'}{T_1'} = \frac{(2\,\kappa\,\text{Ma}_\text{s}^2-\kappa+1)\,(2+(\kappa-1)\,\text{Ma}_\text{s}^2)}{(\kappa+1)^2\,\text{Ma}_\text{s}^2}
            \end{equation*}%
            \vspace{2mm}
            \begin{equation*}
                \frac{T_{02}'}{T_1} = \frac{2}{\kappa+1}\,\left((\kappa-1)\,\text{Ma}_\text{s}^2+\frac{3-\kappa}{2}\right)
            \end{equation*}
            \\
            \rule{\dimexpr\cellwidth-2\TileSep}{0.5pt}\\
            \titlerm{$\text{Ma}_\text{s} > 7$}\\ \vspace{2mm}
            \resizebox{\cellwidth-4.5mm}{!}{
                \begin{tabularx}{\cellwidth+2mm}{Z|Z}
                    $\text{Ma}_2 \approx \sqrt{\dfrac{\kappa-1}{2\,\kappa}}$ & $\text{Ma}_2' \approx \sqrt{\dfrac{2}{\kappa\,(\kappa-1)}}$ \\[3.0ex]
                    $\dfrac{p_{02}}{p_{01}} \approx \left(\dfrac{\kappa+1}{2\,\kappa}\,\left(\dfrac{\kappa+1}{\kappa-1}\,\kappa\right)\,\dfrac{1}{\text{Ma}_\text{s}^2}\right)^\frac{1}{\kappa-1}$ & $\dfrac{p_{02}'}{p_{01}'} = \dfrac{p_{02}'}{p_1'} = \dfrac{p_{02}'}{p_1} \approx \left(\dfrac{\kappa+1}{\kappa}\right)^\frac{1}{\kappa-1}\text{Ma}_\text{s}^2$ \\[4.5ex]
                    $\dfrac{T_2}{T_1} \approx 2\,\kappa\,\dfrac{(\kappa-1)}{(\kappa+1)^2}\,\text{Ma}_\text{s}^2$ & $\dfrac{T_{02}'}{T_{01}'} = \dfrac{T_{02}'}{T_1'} = \dfrac{T_{02}'}{T_1} \approx 2\,\dfrac{(\kappa-1)}{(\kappa+1)}\,\text{Ma}_\text{s}^2$ \\[-0.5ex]
                     & 
                \end{tabularx}
            }
            \\ \vspace{-1mm}
            \begin{equation*}
                \frac{p_2}{p_1} = \frac{p_2'}{p_1'} \approx \frac{2\,\kappa}{\kappa+1}\,\text{Ma}_\text{s}^2
            \end{equation*}
        }{0}{1}{0}{0}

    \setlength{\TileSep}{2mm}

    % B3) Zweite Kachel relativ zu A3
    \setlength{\cellwidth}{80mm}
    \Tile{B3}{(A3.north east)}{\cellwidth}{\rowheight}
        {
            \titlebf{Düsenströmung}\\ \vspace{4mm}
            \setlength{\TileSep}{0mm}
            \begin{minipage}{0.44\cellwidth}
                \centering
                \includegraphics[width=\textwidth]{duesenstroemung-beispiel.pdf}\\ \vspace{-5.15mm}
                \includegraphics[width=\textwidth]{duesenstroemung-beispiel-druckdiagramm.pdf}\\ \vspace{-7.29mm}
                \includegraphics[width=\textwidth]{duesenstroemung-beispiel-machdiagramm.pdf}
            \end{minipage}%
            \hspace{4mm}
            \begin{minipage}{0.44\cellwidth}
                \centering
                \includegraphics[width=\textwidth]{duesenstroemung-fall-a.pdf}\\
                \includegraphics[width=\textwidth]{duesenstroemung-fall-b.pdf}\\
                \includegraphics[width=\textwidth]{duesenstroemung-fall-c.pdf}\\
                \includegraphics[width=\textwidth]{duesenstroemung-fall-d.pdf}\\
                \includegraphics[width=\textwidth]{duesenstroemung-fall-e.pdf}\\
                \includegraphics[width=\textwidth]{duesenstroemung-fall-f.pdf}\\
                \includegraphics[width=\textwidth]{duesenstroemung-fall-g.pdf}
            \end{minipage}
            \setlength{\TileSep}{2mm}
        }{0}{0}{0}{0}

\end{tikzpicture}
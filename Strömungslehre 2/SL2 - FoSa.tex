\documentclass[a4paper,10pt]{article}

\usepackage[top=10mm, left=0mm, right=0mm, bottom=0mm]{geometry} % Für das Blatt-Layout
\usepackage{tikz}                                                % Für die Grafik-Erstellung
\usetikzlibrary{arrows.meta, positioning}                        % Für Pfeilspitzen & relative Positionierung
\usepackage{amsmath, amssymb, esint}                             % Mathematik-Umgebung
\usepackage{graphicx}                                            % Für Bilder
\usepackage{hyperref}                                            % Für Links
\usepackage{float}                                               % Für einfachere Float-Optionen, bspw. [H] an figures
\usepackage{tabularx}                                            % Für mehr Tabellen-Tools
\usepackage{mathtools}                                           % Mehr Tricks in der Mathe-Umgebung
\usepackage[dvipsnames]{xcolor}                                  % Für mehr Farben-Presets
\usepackage{contour}                                             % Zur Einstellung verschiedener Konturabstände
\usepackage[normalem]{ulem}                                      % Für anliegende Unterstreichungen von Titeln
\usepackage{fancyhdr}                                            % Für eigene Kopf- und Fußzeilen

\contourlength{0.8pt}                                            % Abstand der Unterstreichungen zu herabhängenden Buchstaben einstellen
\newcommand{\myul}[1]{%                                          % Eigene Unterstreichungen definieren
    \uline{\phantom{#1}}%
    \llap{\contour{white}{#1}}%
}
\graphicspath{{./images/}}                                       % Pfad für Bilder
\usetikzlibrary{calc}                                            % Für ([xshift=..]node.anchor)
\fancyhead[L,C,R]{}                                              % Header leer einstellen
\renewcommand{\headrulewidth}{0pt}                               % Header-Trennlinie entfernen
\fancyfoot[L]{}                                                  % Footer links leer einstellen
\fancyfoot[R]{                                                   % Footer rechts mit Seitenzahl
    \vspace*{-22mm}\\
    \thepage\hspace*{10mm}
}                         
\fancyfoot[C]{                                                   % Hinweis im Footer
    \vspace*{-20mm}\\
    \tiny Gleichungsnummern aus der Vorlesung von Prof.\ Dr.-Ing. Krämer {\color{NavyBlue}in Blau}. CAS-Programmaufrufe {\color{RedOrange}in Orange}.
}
\newcolumntype{Z}{>{\centering\arraybackslash}X} % Eigenen Spaltentyp in Tabellen definieren (Auto-Größe & zentriert)

\renewcommand{\ULdepth}{1.8pt}

% ---- Kanten-Helfer (wähle die zu ziehenden Kanten aus)
\newcommand{\topedge}[1]{\draw[line width=0.6pt]   (#1.north west)--(#1.north east);}
\newcommand{\bottomedge}[1]{\draw[line width=0.6pt](#1.south west)--(#1.south east);}
\newcommand{\leftedge}[1]{\draw[line width=0.6pt]  (#1.north west)--(#1.south west);}
\newcommand{\rightedge}[1]{\draw[line width=0.6pt] (#1.north east)--(#1.south east);}

% ---- Innenabstand als Länge anlegen
\newlength{\TileSep}
\setlength{\TileSep}{2mm}

% ---- Spaltenhöhe und Zellbreite als Länge anlegen
\newlength{\rowheight}
\newlength{\cellwidth}

% ---- Funktion "Tile": eine Kachel in einem Rutsch anlegen & optionale Kanten
% \Tile{Name}{Koordinate}{Breite}{Höhe}{Inhalt}{Top}{Right}{Bottom}{Left}  (0/1-Schalter)
\newcommand{\Tile}[9]{%
    \node[
        anchor=north west, inner sep=\TileSep, minimum width=#3, minimum height=#4
    ] (#1) at #2
    {%
        % Parbox mit exakt (Höhe - 2 * Innenabstand) und (Breite - 2 * Innenabstand)
        % [t][...][t]  -> Inhalt an den oberen Rand der Box
        \parbox[t][\dimexpr#4-2\TileSep\relax][t]{\dimexpr#3-2\TileSep\relax}{\centering\small #5}%
    };%
    \ifnum#6=1 \topedge{#1}\fi
    \ifnum#7=1 \rightedge{#1}\fi
    \ifnum#8=1 \bottomedge{#1}\fi
    \ifnum#9=1 \leftedge{#1}\fi
}

% ---- Funktionen für Titel mit korrekter Größe (nur die Formeln sollen platzsparend klein sein)
\newcommand{\titlebf}[1]{%
    \myul{\textbf{\normalsize{#1}}}
}
\newcommand{\titlerm}[1]{%
    \myul{\normalsize{#1}}
}

% ---------------------------------------------------------------------------- %

% Titel der Formelsammlung hier ändern (bis zu zwei Zeilen möglich):
\newcommand{\thetitle}{\textbf{
    Strömungslehre II
}}

% ---------------------------------------------------------------------------- %

\begin{document}
\pagestyle{fancy}

% Deckblatt einbinden
\thispagestyle{empty}

\centering

\vspace*{5mm}
\includegraphics[width=0.7\textwidth]{FLURUS-Logo.pdf}

\vspace{65mm}
\begin{minipage}[c][30mm][c]{\textwidth}
    \centering
    \Huge\thetitle
\end{minipage}

\vspace{30mm}
{\Huge\textbf{Formelsammlung}}

\vspace{78mm}
{\large Ein Open-Source-Projekt der Fachschaft}

\vspace*{3mm}
\texttt{
    \tiny Bei Fragen, Fehlerkorrekturen oder Anmerkungen kontaktiert gerne die Fachschaft. Der Quellcode dieser Formelsammlung steht auf \href{https://github.com/m4zeRobot/Project-FoSa}{GitHub} zur Verfügung. \\[-2.0ex]
}

\vspace{3mm}
\texttt{
    \tiny Die Inhalte in diesem Dokument werden Studenten der Luft- und Raumfahrttechnik an der Universität Stuttgart im Rahmen des Studiums der Luft- und Raumfahrttechnik an der \\
    Universität Stuttgart zur Verfügung gestellt. Diese dürfen ausschließlich für akademische Zwecke verwendet werden und sind Studenten der Luft- und Raumfahrttechnik an der \\
    Universität Stuttgart vorbehalten. Weder Korrektheit noch Vollständigkeit der Inhalte wird gewährleistet und weder für fehlerhafte noch für fehlende Informationen wird \\
    gehaftet. Die Verwendung verläuft auf eigene Gefahr und wird nicht empfohlen. Für jegliche Folgen die aus der Verwendung der in dieser Formelsammlung enthaltenen Formeln, \\
    Grafiken und Informationen hervorgehen ist der Anwender verantwortlich. Vervielfältigung dieses Dokumentes ohne explizite Einverständniserklärung der Autoren der verwendeten \\
    grafischen und textbasierten Inhalte ist rechtswidrig. \\[-2.0ex]
}
\rule{0.86\textwidth}{0.5pt} \\ \vspace*{1mm}
\normalsize
\begin{tabular}{l c r}
    \normalsize
    Pfaffenwaldring 27\hspace{26mm} & Tel.: \href{tel:+4971168562319}{(+49) 711 685 - 6 23 19} & \hspace{33mm}\href{https://www.flurus.de/}{www.flurus.de} \\
    70569 Stuttgart                 & Fax: (+49) 711 685 - 6 20 39                             &              \href{mailto:info@flurus.de}{info@flurus.de}
\end{tabular}

\newpage

% ---- Ab hier beginnt die eigentliche Formelsammlung

% Diese Seite verwendet eine eigene Seitengeometrie und verzichtet auf die Fußnote.
\newgeometry{left=2cm, right=2cm, top=1cm, bottom=1cm}
\thispagestyle{empty}

% Überschrift der Seite, zentriert
\begin{center}
    \section*{\underline{Theoretische Grundlagen}}
\end{center}

\begin{raggedright}

\subsection*{\myul{Terminologie}} \vspace*{-2mm}
\begin{table}[H]
    \begin{tabular}{l l l}
        \textbf{Begriff} & \textbf{Bedeutung} & \textbf{Mathematisch} \\[2.0ex]
        Stromlinie   & \begin{tabular}[c]{@{}l@{}}Linie, die zu jedem Zeitpunkt tangential zur Geschwindigkeit\\ verläuft (siehe Strömung I)\end{tabular} & $\Psi =$ konstant \\[3.0ex]
        Wirbellinie  & \begin{tabular}[c]{@{}l@{}}Laufen zu einer bestimmten Zeit an jedem Ort tangential zum\\ Wirbel- oder Drehvektor $\underline{\omega}$ (analog zur Stromlinie).\end{tabular} & - \\[2.0ex]
        Wirbelfaden  & \begin{tabular}[c]{@{}l@{}}Gesamtheit der Wirbellinien, die durch eine Fläche $A_1$ ein-\\ und eine Fläche $A_2$ austreten.\end{tabular} & - \\[2.0ex]
        Wirbelröhre  & \begin{tabular}[c]{@{}l@{}}Wirbellinien durch die Punkte einer geschlossenen Kurve bilden\\ die Mantelfläche der Wirbelröhre.\end{tabular} & - \\[2.0ex]
        Wirbelstrom  & Analog zum Volumenstrom. & $\underline{\omega}\cdot\underline{A} =$ konstant \\[3.0ex]
        Isobare      & Linie gleicher statischer Drücke in der Strömung. & $p =$ konstant \\[3.0ex]
        Isotache     & Linie gleicher Geschwindigkeit in der Strömung. & $\left|\underline{v}\right| =$ konstant \\[3.0ex]
        Isokline     & Linie gleicher Stromlinienneigung in der Strömung. & $\tan(\alpha) = \dfrac{v}{u} =$ konstant \\
    \end{tabular}
\end{table}

\end{raggedright}

\restoregeometry
\restoregeometry

\setcounter{page}{1}
\setcounter{page}{1}
\begin{center}
    \section*{\underline{Wirbeltheorie}}
\end{center}

\begin{tikzpicture}[remember picture, overlay]
% ---- ERSTE REIHE: Am Seitenursprung starten
\setlength{\rowheight}{54mm} % Erste Reihenhöhe: 54mm

    % A1) Erste Kachel absolut platzieren (vom Seiten-Nordwest-Anker aus)
    \setlength{\cellwidth}{62mm} % Breite einstellen
    \Tile{A1}{([xshift=10mm,yshift=-20mm]current page.north west)}{\cellwidth}{\rowheight}
        {
            \titlebf{Drehungsfrei}
            \begin{equation*}
                \text{rot}(\underline{v}_1) = 0\ \ |\ \ \text{div}(\rho\underline{v}_1) \neq 0\ \ |\ \ \underline{v}_1 = \nabla(\phi)
            \end{equation*}
            $\phi =$ Skalares Geschwindigkeitspotential \\[3.0ex]
            % Unterteilung des folgenden Bereiches in 2 Spalten mit tabularx
            \begin{tabularx}{\dimexpr\cellwidth-2\TileSep}{c|c} % Breite der tabularx = (Zellenbreite - 2 * Tile-Randabstand)
                \begin{minipage}{0.4\cellwidth}
                    \includegraphics[width=0.4\cellwidth]{wirbeltheorie-drehungsfrei.pdf}
                \end{minipage} & \begin{tabular}[c]{@{}c@{}}
                    \titlebf{Volumenstrom}\\[1.5ex] $Q=\underline{v}\cdot \underline{A}$
                \end{tabular}
            \end{tabularx}
        }{0}{1}{1}{0} % Kanten: unten & rechts

    % B1) Zweite Kachel mitte
    \setlength{\cellwidth}{68mm} % Breite einstellen
    \Tile{B1}{(A1.north east)}{\cellwidth}{\rowheight}
        {
            \titlebf{Drehungsbehaftet}
            \begin{equation*}
                \text{rot}(\underline{v}_2) \neq 0\ \ |\ \ \text{div}(\rho\underline{v}_2) = 0\ \ |\ \ \rho\underline{v}_2 = \rho_\text{b}\text{rot}(\psi)
            \end{equation*}
            $\psi =$ Vektorielles Wirbelpotential \\ \vspace{3mm} % 2mm Platz über dem kommenden Bild
            % Unterteilung des folgenden Bereiches in 2 Spalten mit tabularx
            \begin{tabularx}{\dimexpr\cellwidth-2\TileSep}{c|c} % Breite der tabularx = (Zellenbreite - 2 * Tile-Randabstand)
                \begin{minipage}{0.44\cellwidth}
                    \includegraphics[width=0.44\cellwidth]{wirbeltheorie-wirbel.pdf}
                \end{minipage} & \hspace{2mm}\begin{tabular}[c]{@{}c@{}}
                    \titlebf{Wirbelstrom}\\[1.0ex]
                    $\underline{\omega}\cdot \underline{A}$ \\[2.0ex]
                    \titlebf{Wirbelstärke}\\[1.0ex]
                    $\underline{\omega}=\dfrac{1}{2}\text{rot}(\underline{v})$
                \end{tabular}
            \end{tabularx}
        }{0}{1}{1}{0} % Kanten: unten & rechts

    % C1) Dritte Kachel rechts
    \setlength{\cellwidth}{60mm} % Breite einstellen (Summe aller Zellen sei 190mm!)
    \Tile{C1}{(B1.north east)}{\cellwidth}{\rowheight}
        {
            \titlebf{Zirkulation} {\normalsize(positiv: $\mathrlap{\circlearrowleft}\hspace*{0.88mm}\cdot$ )}
            \begin{align}
                \Gamma &= \oint_S \underline{v}\cdot \text{d}\underline{s} = \oint_S v \cos(\alpha)\text{d}s \tag*{\color{NavyBlue}(7.20a)}\\
                       &= \oiint_A \text{rot}(\underline{v})\cdot\underline{n}\text{d}A = \oiint_A\underline{\Omega}\cdot \underline{n}\text{d}A \nonumber
            \end{align} \\[-0.5ex]
            \rule{\dimexpr\cellwidth-2\TileSep}{0.5pt} \\
            \titlebf{Laplace-Operator} \\[-3.0ex]
            \begin{equation*}
                \Delta(\underline{\psi}) = 
                \begin{pmatrix}
                    \tfrac{\partial^2\psi_x}{\partial x^2} + \tfrac{\partial^2\psi_x}{\partial y^2} + \tfrac{\partial^2\psi_x}{\partial z^2} \\[1.0ex]
                    \tfrac{\partial^2\psi_y}{\partial x^2} + \tfrac{\partial^2\psi_y}{\partial y^2} + \tfrac{\partial^2\psi_y}{\partial z^2} \\[1.0ex]
                    \tfrac{\partial^2\psi_z}{\partial x^2} + \tfrac{\partial^2\psi_z}{\partial y^2} + \tfrac{\partial^2\psi_z}{\partial z^2}
                \end{pmatrix}
            \end{equation*}
        }{0}{0}{1}{0} % Kante: unten


% --- ZWEITE REIHE: unter A1 starten, dann wieder nach rechts anbauen
\setlength{\rowheight}{28mm} % Zweite Reihenhöhe

    % X2) EIGENES FELD für die Zeilen-Überschrift
    %     (Volle Breite: 190mm + 2*10mm Seitenrand = 210mm von DIN A4)
    %     Relativ zu Feld A1 (untere linke Ecke), am Boden der Kachel X2, 2mm nach oben
    \Tile{X2}{([yshift=-3mm]A1.south west)}{190mm}{10mm}
        {
            \titlebf{Rotation}
        }{0}{0}{0}{0} % Kanten: unten & rechts

    % A2) Erste Kachel relativ zu A1
    \setlength{\cellwidth}{35mm} % Breite einstellen
    \Tile{A2}{([yshift=-10mm]A1.south west)}{\cellwidth}{\rowheight}
        {
            \titlebf{Allgemein} \\[-1.5ex]
            \begin{equation*}
                \underline{\omega} = 2\underline{\omega} = \text{rot}(\underline{v})
            \end{equation*}
            {\color{NavyBlue}(7.11)}
        }{0}{1}{1}{0} % Kanten: unten & rechts

    % B2) Zweite Kachel relativ zu A2
    \setlength{\cellwidth}{65mm} % Breite einstellen
    \Tile{B2}{(A2.north east)}{\cellwidth}{\rowheight}
        {
            \titlebf{Dichtebeständig} \\[3.0ex]
            \begin{tabularx}{\dimexpr\cellwidth-2\TileSep}{c|c} % Breite der tabularx = (Zellenbreite - 2 * Tile-Randabstand)
                \centering
                \begin{tabular}[c]{@{}c@{}}\underline{Allgemein}\\[3.0ex]$\underline{\Omega} = \nabla(\text{div}(\underline{\psi})) - \Delta(\underline{\psi})$\\[3.0ex]\end{tabular} & \begin{tabular}[c]{@{}c@{}}\underline{Ebene Platte}\\[3.0ex]$\Omega_z = -\Delta(\psi_z)$\\[3.0ex]\end{tabular}
            \end{tabularx}
        }{0}{1}{1}{0} % Kanten: unten & rechts

    % C2) Dritte Kachel relativ zu B2
    \setlength{\cellwidth}{40mm} % Breite einstellen
    \Tile{C2}{(B2.north east)}{\cellwidth}{\rowheight}
        {
            \titlebf{Kartesisch} \\[-2.0ex]
            \begin{equation*}
                \underline{\Omega} =
                \begin{pmatrix}
                    \tfrac{\partial w}{\partial y} - \tfrac{\partial v}{\partial z} \\[1.0ex]
                    \tfrac{\partial u}{\partial z} - \tfrac{\partial w}{\partial x} \\[1.0ex]
                    \tfrac{\partial v}{\partial x} - \tfrac{\partial u}{\partial y}
                \end{pmatrix}
            \end{equation*}
        }{0}{1}{1}{0} % Kante: unten & rechts

    % D2) Dritte Kachel relativ zu C2
    \setlength{\cellwidth}{50mm} % Breite einstellen
    \Tile{D2}{(C2.north east)}{\cellwidth}{\rowheight}
        {
            \titlebf{Zylindrisch} \\[-2.5ex]
            \begin{equation*}
                \underline{\Omega} =
                \begin{pmatrix}
                    \tfrac{1}{r}\tfrac{\partial v_z}{\partial\varphi} - \tfrac{\partial v_\varphi}{\partial z} \\[1.0ex]
                    \tfrac{\partial v_r}{\partial z} - \tfrac{\partial v_z}{\partial r} \\[1.0ex]
                    \tfrac{1}{r}\left(\tfrac{\partial}{\partial r}\left(r\cdot v_\varphi\right) - \tfrac{\partial v_r}{\partial\varphi}\right)
                \end{pmatrix}
            \end{equation*}
        }{0}{0}{1}{0} % Kante: unten

% --- DRITTE REIHE: unter A2 starten, dann wieder nach rechts anbauen
\setlength{\rowheight}{50mm} % Dritte Reihenhöhe

    % X3) EIGENES FELD für die Zeilen-Überschrift
    \Tile{X3}{([yshift=-3mm]A2.south west)}{190mm}{10mm}
        {
            \titlebf{Wirbelsätze}
        }{0}{0}{0}{0} % Kanten: unten & rechts

    % A3) Erste Kachel relativ zu A2
    \setlength{\cellwidth}{95mm} % Breite einstellen
    \Tile{A3}{([yshift=-10mm]A2.south west)}{\cellwidth}{\rowheight}
        {
            \titlebf{1. Helmholtz'scher Wirbelsatz}
            \begin{equation*}
                \text{div}(\underline{\omega}) = \text{div}(\underline{\Omega}) = 0
            \end{equation*} \\[-1.5ex]
            \rule{\dimexpr\cellwidth-2\TileSep}{0.5pt} \\[2.5ex]
            \titlebf{2. Helmholtz'scher Wirbelsatz} \\[3.0ex]
            \begin{tabularx}{\dimexpr\cellwidth-2\TileSep}{p{0.37\cellwidth}|X}
                \begin{tabular}[c]{@{}c@{}}
                    \myul{Reibungsfrei \& Barotrop} \\[3.0ex]
                    $\dfrac{\text{D}}{\text{D}t}\left(\dfrac{\underline{\omega}}{\rho}\right)=\dfrac{\underline{\omega}}{\rho}\cdot\nabla\left(\underline{v}\right)$
                \end{tabular} & \begin{tabular}[c]{@{}c@{}} \\[-3.5ex]
                    \myul{2D / Rotationssym. \& Inkompr.} \\[3.0ex]
                    $\dfrac{\text{D}\omega}{\text{D}t}=\dfrac{\partial\omega}{\partial t}+\underline{v}\cdot\nabla(\omega)=0$
                \end{tabular} \\[6.0ex]
            \end{tabularx}
        }{0}{1}{1}{0} % Kanten: unten & rechts

    % B3) Zweite Kachel relativ zu A3
    \setlength{\cellwidth}{50mm} % Breite einstellen
    \Tile{B3}{(A3.north east)}{\cellwidth}{\rowheight}
        {
            \titlebf{Wirbeltransportgleichung} \\[2.0ex]
            \titlerm{Allgemein} \\[-2.0ex]
            \begin{equation*}
                \frac{\text{D}}{\text{D}t}\left(\underline{\omega}\right)=\underline{\omega}\cdot\nabla\left(\underline{v}\right)+\nu\Delta\left(\underline{\omega}\right)
            \end{equation*} \\
            \rule{\dimexpr\cellwidth-2\TileSep}{0.5pt} \\[2.0ex]
            \titlerm{2D / Rotationssym.} \\[-2.0ex]
            \begin{equation*}
                \frac{\text{D}\omega}{\text{D}t}=\frac{\partial\omega}{\partial t} + \underline{v}\cdot\nabla(\omega) = 0
            \end{equation*}
        }{0}{1}{1}{0} % Kanten: unten & rechts

    % C3) Zweite Kachel relativ zu B3
    \setlength{\cellwidth}{45mm} % Breite einstellen
    \Tile{C3}{(B3.north east)}{\cellwidth}{\rowheight}
        {
            \titlebf{Satz von}\\\titlebf{Thomson / Kelvin} \\[1.0ex]
            \begin{equation*}
                \frac{\text{D}\Gamma}{\text{D}t} = - \oint\limits_{S(t)} \left(\frac{1}{\rho}\right)\text{d}p
            \end{equation*} \\[1.0ex]
            Reibungsfrei,\\
            Barotrop / Inkompressibel,\\
            $U$ existiert
        }{0}{0}{1}{0} % Kanten: unten & rechts


% --- VIERTE REIHE: unter A3, dann rechts anbauen
\setlength{\rowheight}{44mm}

    % X4) Zeilen-Überschrift
    \Tile{X4}{([yshift=-3mm]A3.south west)}{190mm}{10mm}
        {
            \hspace*{34mm}\titlebf{Biot-Savart'sches Gesetz}\ \ \ {\normalsize(positiv: $\mathrlap{\circlearrowleft}\hspace*{0.88mm}\cdot$\ ,\ RHR)}
        }{0}{0}{0}{0}

    % A4) Erste Kachel relativ zu A3
    \setlength{\cellwidth}{38mm}
    \Tile{A4}{([yshift=-10mm]A3.south west)}{\cellwidth}{\rowheight}
        {
            \titlebf{Vektorform}
            \begin{equation*}
                \text{d}\underline{v} = \frac{\Gamma}{4\pi}\cdot\frac{\underline{a}\times\text{d}\underline{s}}{a^3}
            \end{equation*} \\[-0.5ex]
            \rule{\dimexpr\cellwidth-2\TileSep}{0.5pt} \\[1.5ex]
            \titlebf{Betragsform}
            \begin{equation*}
                \text{d}v = \frac{\Gamma}{4\pi}\cdot\frac{\sin(\alpha)\ \text{d}s}{a^2}
            \end{equation*}
        }{0}{1}{1}{0}

    % B4) Zweite Kachel relativ zu A4
    \setlength{\cellwidth}{42mm}
    \Tile{B4}{(A4.north east)}{\cellwidth}{\rowheight}
        {
            \titlebf{Einseiting Unendlich}
            \begin{equation*}
                v = \frac{\Gamma}{4\pi r}\cdot\left(1+\cos(\alpha_1)\right)
            \end{equation*}
            \begin{equation*}
                \alpha_2 = \pi
            \end{equation*} \\[-1.0ex]
            \rule{\dimexpr\cellwidth-2\TileSep}{0.5pt} \\[1.5ex]
            \titlebf{Gerade Wirbellinie}
            \begin{equation*}
                \text{d}v = \frac{\Gamma}{4\pi r}\cdot\sin(\alpha)\ \text{d}\alpha
            \end{equation*}
        }{0}{1}{1}{0}

    % C4) Dritte Kachel relativ zu B4
    \setlength{\cellwidth}{28mm}
    \Tile{C4}{(B4.north east)}{\cellwidth}{\rowheight}
        {
            \vspace{10mm}
            \titlebf{Beidseitig} \\
            \ \titlebf{Unendlich}
            \begin{equation*}
                v = \frac{\Gamma}{2\pi r}
            \end{equation*}
        }{0}{1}{1}{0}

    % D4) Vierte Kachel relativ zu C4
    \setlength{\cellwidth}{82mm}
    \Tile{D4}{(C4.north east)}{\cellwidth}{\rowheight}
        {
            \vspace{5mm}
            \begin{tabularx}{\dimexpr\cellwidth-2\TileSep}{X X}
                \hspace{1mm}\includegraphics[height=0.65\rowheight]{biot-savart-1.pdf} & \hspace{-6mm}\includegraphics[height=0.65\rowheight]{biot-savart-2.pdf}
            \end{tabularx}
        }{0}{0}{1}{0}

% --- FÜNFTE REIHE: unter A4, dann rechts anbauen
\setlength{\rowheight}{46mm}

    % X5) Zeilen-Überschrift
    \Tile{X5}{([yshift=-3mm]A4.south west)}{190mm}{10mm}
        {
            \titlebf{arctan2 - Funktion}
        }{0}{0}{0}{0}

    % A5) Erste Kachel relativ zu A3
    \setlength{\cellwidth}{70mm}
    \Tile{A5}{([yshift=-10mm]A4.south west)}{\cellwidth}{\rowheight}
        {
            \vspace{-1mm}
            {\color{RedOrange}arctan2($x,y$)}
            \begin{equation*}
                =
                \begin{cases}
                    \tan^{-1}\left(\tfrac{y}{x}\right)     & \text{für }x>0      \\[0.5ex]
                    \tan^{-1}\left(\tfrac{y}{x}\right)+\pi & \text{für }x<0, y>0 \\[0.5ex]
                    \pm\pi                                 & \text{für }x<0, y=0 \\[0.5ex]
                    \tan^{-1}\left(\tfrac{y}{x}\right)-\pi & \text{für }x<0, y<0 \\[0.5ex]
                    +\tfrac{\pi}{2}                        & \text{für }x=0, y>0 \\[0.5ex]
                    -\tfrac{\pi}{2}                        & \text{für }x=0, y<0 \\[0.5ex]
                    \text{undefiniert}                     & \text{für }x=0, y=0
                \end{cases}
            \end{equation*}
        }{0}{1}{0}{0}

% B5) Zweite Kachel relativ zu A4
    \setlength{\cellwidth}{120mm}
    \Tile{B5}{(A5.north east)}{\cellwidth}{\rowheight}
        {
            \vspace{3mm}
            \titlerm{Rechenregeln}
            \begin{equation*}
                \text{arctan2}(x,-y) = -\text{arctan2}(x,y)
            \end{equation*}
            \begin{equation*}
                \text{arctan2}(-x,y) =
                \begin{cases}
                    -\text{arctan2}(-x,y) + {\color{RedOrange}\text{sign}(y)}\cdot\pi & \text{für }y\neq 0        \\[0.8ex]
                    0                                                                 & \text{für }y=0 \wedge x<0 \\[0.8ex]
                    \pm\pi                                                            & \text{für }y=0 \wedge x>0
                \end{cases}
            \end{equation*} \\[1.0ex]
            (Die Funktion sign($x$) gibt das Vorzeichen von $x$ zurück.)
        }{0}{0}{0}{0}

\end{tikzpicture}

% ### Erste Seite der Potentialtheorie ###
\begin{center}
    \section*{\underline{Potentialtheorie}}
\end{center}

\begin{tikzpicture}[remember picture, overlay]

% --- ERSTE REIHE: Am Seitenursprung anfangen
\setlength{\rowheight}{20mm}

    % A1) Erste Kachel relativ zum Seitenursprung
    \setlength{\cellwidth}{70mm}
    \Tile{A1}{([xshift=10mm, yshift=-20mm]current page.north west)}{\cellwidth}{\rowheight}
        {
            \vspace{-2mm}
            \titlebf{Geschwindigkeitspotential} \\[1.5ex]
            \begin{tabularx}{\dimexpr\cellwidth-2\TileSep}{Z|Z}
                \begin{tabular}[c]{@{}c@{}}
                    \myul{Definition} \\[2.0ex]
                    $\underline{v} = \nabla(\phi)$ \vspace{2mm}
                \end{tabular} & \begin{tabular}[c]{@{}c@{}}
                    \myul{Inkompressibel} \\[2.0ex]
                    $\Delta(\phi) = 0$ \vspace{2mm}
                \end{tabular}
            \end{tabularx}
        }{0}{1}{1}{0}

    % B1) Zweite Kachel relativ zu A1
    \setlength{\cellwidth}{62mm}
    \Tile{B1}{(A1.north east)}{\cellwidth}{\rowheight}
        {
            \vspace{-2mm}
            \titlebf{Grundgleichung} \\[-1.5ex]
            \begin{equation*}
                \frac{\partial\phi}{\partial t} +\frac{1}{2} \left(\nabla(\phi)\right)^2 + \frac{r}{\rho} + U = F(t)
            \end{equation*}
        }{0}{1}{1}{0}

    % C1) Zweite Kachel relativ zu B1
    \setlength{\cellwidth}{58mm}
    \Tile{C1}{(B1.north east)}{\cellwidth}{\rowheight}
        {
            \vspace{-2mm}
            \titlebf{Zirkulation} \\[-1.5ex]
            \begin{equation*}
                \Gamma_{1\rightarrow 2} = \int_1^2 \underline{v}\cdot\text{d}\underline{s} = \phi_2 - \phi_1
            \end{equation*}
        }{0}{0}{1}{0}

% --- Zweite REIHE: direkt an erster Reihe
\setlength{\rowheight}{26mm}

    % A2) Erste Kachel relativ zu A1
    \setlength{\cellwidth}{98mm}
    \Tile{A2}{(A1.south west)}{\cellwidth}{\rowheight}
        {
            \titlebf{Laplace-Gleichung (ausgeschrieben)} \\[1.5ex]
            \begin{tabularx}{\dimexpr\cellwidth-2\TileSep}{c|c}
                \begin{tabular}[c]{@{}c@{}}
                    \myul{Kartesisch} \\[2.0ex]
                    $\dfrac{\partial^2 \phi}{\partial x^2} + \dfrac{\partial^2 \phi}{\partial y^2} + \dfrac{\partial^2 \phi}{\partial z^2} = 0$ \vspace{2mm}
                \end{tabular}\hspace{1mm} & \begin{tabular}[c]{@{}c@{}}
                    \myul{Zylinderkoordinaten} \\[2.0ex]
                    \hspace{2mm}$\dfrac{1}{r} \dfrac{\partial}{\partial r} \left( r \dfrac{\partial \phi}{\partial r} \right) + \dfrac{1}{r^2} \dfrac{\partial^2 \phi}{\partial \varphi^2} + \dfrac{\partial^2 \phi}{\partial z^2} = 0$ \vspace{2mm}
                \end{tabular}
            \end{tabularx}
        }{0}{1}{1}{0}

    % B2) Zweite Kachel relativ zu A2
    \setlength{\cellwidth}{92mm}
    \Tile{B2}{(A2.north east)}{\cellwidth}{\rowheight}
        {
            \titlebf{Randbedingungen} \\[1.5ex]
            \begin{tabularx}{\dimexpr\cellwidth-2\TileSep}{Z|Z|Z}
                \begin{tabular}[c]{@{}c@{}}
                    \myul{Feste Wand} \\[2.0ex]
                    $\dfrac{\partial\phi}{\partial n} = 0$ \vspace{3.8mm}
                \end{tabular} & \begin{tabular}[c]{@{}c@{}}
                    \myul{Bewegte Wand} \\[2.0ex]
                    $\dfrac{\partial\phi}{\partial n} = \underline{u}\cdot\underline{n}$ \vspace{3.8mm}
                \end{tabular} & \begin{tabular}[c]{@{}c@{}}
                    \myul{Im Unendlichen} \\[2.0ex]
                    $\nabla(\phi)=\underline{v}_\infty$ \vspace{3.8mm}
                \end{tabular}
            \end{tabularx}
        }{0}{0}{1}{0}

% --- Dritte REIHE: unter A2, dann rechts anbauen
\setlength{\rowheight}{36mm}

    % A3) Erste Kachel relativ zu A2
    \setlength{\cellwidth}{95mm}
    \Tile{A3}{(A2.south west)}{\cellwidth}{\rowheight}
        {
            \titlebf{Ebene Geschwindigkeiten} \\[1.5ex]
            \begin{tabularx}{\dimexpr\cellwidth-2\TileSep}{Z|Z}
                \begin{tabular}[c]{@{}c@{}}
                    \myul{Kartesisch} \\[2.5ex]
                    $u = \dfrac{\partial\phi}{\partial x} = \dfrac{\partial\psi}{\partial y}$ \\[3.0ex]
                    $v = \dfrac{\partial\phi}{\partial y} = -\dfrac{\partial\psi}{\partial x}$ \vspace*{2mm}
                \end{tabular} & \begin{tabular}[c]{@{}c@{}}
                    \myul{Kartesisch} \\[2.5ex]
                    $u_r = \dfrac{\partial\phi}{\partial r} = \dfrac{1}{r}\cdot\dfrac{\partial\psi}{\partial\varphi}$ \\[3.0ex]
                    $v_\varphi = \dfrac{1}{r}\cdot\dfrac{\partial\phi}{\partial\varphi} = -\dfrac{\partial\psi}{\partial r}$ \vspace*{2mm}
                \end{tabular}
            \end{tabularx}
        }{0}{1}{1}{0}

    % B3) Zweite Kachel relativ zu A3
    \setlength{\cellwidth}{95mm}
    \Tile{B3}{(A3.north east)}{\cellwidth}{\rowheight}
        {
            \titlebf{Geschwindigkeitsbeziehungen} \\[1.5ex]
            \begin{tabularx}{\dimexpr\cellwidth-2\TileSep}{Z|Z}
                \begin{tabular}[c]{@{}c@{}}
                    \myul{Eben (Cauchy-Riemann)} \\[2.5ex]
                    $u = \dfrac{\partial\phi}{\partial x} = \dfrac{\partial\psi}{\partial y}$ \\[3.0ex]
                    $v = \dfrac{\partial\phi}{\partial y} = -\dfrac{\partial\psi}{\partial x}$ \vspace*{2mm}
                \end{tabular} & \begin{tabular}[c]{@{}c@{}}
                    \myul{Rotationssymmetrisch} \\[2.5ex]
                    $v_r = \dfrac{\partial\phi}{\partial r} = -\dfrac{1}{r}\cdot\dfrac{\partial\psi}{\partial z}$ \\[3.0ex]
                    $v_z = \dfrac{\partial\phi}{\partial z} = \dfrac{1}{r}\cdot\dfrac{\partial\psi}{\partial r}$ \vspace*{2mm}
                \end{tabular}
            \end{tabularx}
        }{0}{0}{1}{0}

% --- Vierte REIHE: unter A3, dann rechts anbauen
\setlength{\rowheight}{40mm}

    % A4) Erste Kachel relativ zu A3
    \setlength{\cellwidth}{75.5mm}
    \Tile{A4}{(A3.south west)}{\cellwidth}{\rowheight}
        {
            \titlebf{Resultierende Geschw.}
            \begin{equation*}
                F(z) = F(x+iy) = \phi(x,y)+i\psi(x,y)
            \end{equation*} \\[-1.5ex]
            \rule{\dimexpr\cellwidth-2\TileSep}{0.5pt} \\
            \vspace{1mm}
            \begin{tabularx}{\dimexpr\cellwidth-2\TileSep}{Z|Z}
                \begin{tabular}[c]{@{}c@{}}
                    \myul{Geschwindigkeiten} \\[1.5ex]
                    $w = u + iv$ \\[1.5ex]
                    $w_* = u - iv = \dfrac{\text{d}F(z)}{\text{d}z}$ \vspace*{2mm}
                \end{tabular} & \hspace{1mm}\begin{tabular}[c]{@{}c@{}}
                    \myul{Druckverteilung} \\[3.5ex]
                    $p = p_0 - \dfrac{\rho}{2} a^2 r^{2(n-1)}$ \vspace*{6mm}
                \end{tabular}
            \end{tabularx}
        }{0}{1}{1}{0}

    % B4) Zweite Kachel relativ zu A4
    \setlength{\cellwidth}{54.5mm}
    \Tile{B4}{(A4.north east)}{\cellwidth}{\rowheight}
        {
            \titlebf{Ebene Translationsströmung} \\[-1.5ex]
            \begin{equation*}
                F(z) = a\cdot z = (a_1 - i a_2)\cdot z
            \end{equation*}\\[-4.0ex]
            \begin{align*}
                \phi &= a_1 \cdot x + a_2\cdot y \\
                \psi &= a_1 \cdot y - a_2\cdot x
            \end{align*}
            \begin{equation*}
                u=a_1\ ;\ \ \ v=a_2
            \end{equation*}
        }{0}{1}{1}{0}

    % C4) Dritte Kachel relativ zu B4
    \setlength{\cellwidth}{60mm}
    \Tile{C4}{(B4.north east)}{\cellwidth}{\rowheight}
        {
            \titlebf{Ebene Eckenströmung} \\[-2.0ex]
            \begin{equation*}
                F(z) = \frac{a}{n}z^n = \frac{a}{n}r^n (\cos(n\varphi)+i\sin(n\varphi))
            \end{equation*}\\[-4.0ex]
            \begin{align*}
                \phi &= \tfrac{a}{n}r^n\cos(n\varphi) \\[1.0ex]
                \psi &= \tfrac{a}{n}r^n\sin(n\varphi)
            \end{align*}
            \begin{equation*}
                |w| = \left|a\cdot z^{n-1}\right| = |a|\cdot r^{n-1}
            \end{equation*}
        }{0}{0}{1}{0}

% --- Fünfte REIHE: unter A4, dann rechts anbauen
\setlength{\rowheight}{58mm}

    % A5) Erste Kachel relativ zu A4
    \setlength{\cellwidth}{35mm}
    \Tile{A5}{(A4.south west)}{\cellwidth}{\rowheight}
        {
            \titlebf{Feste Wände} \\
            \vspace*{4mm}
            \begin{equation*}
                \varphi_{k,n} = k\cdot\dfrac{\pi}{n}
            \end{equation*}
            \begin{equation*}
                \varepsilon = \dfrac{\pi}{n}
            \end{equation*}
            \begin{equation*}
                \vartheta = \pi - \varepsilon = \pi\cdot\dfrac{n-1}{n}
            \end{equation*}
        }{0}{1}{1}{0}

    % B5) Zweite Kachel relativ zu A5
    \setlength{\cellwidth}{52mm}
    \Tile{B5}{(A5.north east)}{\cellwidth}{\rowheight}
        {
            \titlebf{$\sfrac{1}{2} < n < 1$} \\
            \vspace*{2mm}
            \includegraphics[width=34mm]{images/konvexe_ecke.pdf} \\
            \vspace*{2mm}
            Konvexe Ecke
            \begin{equation*}
                \left|w\right|\sim\dfrac{1}{r^{n-1}}
            \end{equation*}
            $r = 0 \Rightarrow$ Eckenstr. mit $\left|w\right|\rightarrow\infty$
        }{0}{1}{1}{0}

    % C5) Dritte Kachel relativ zu B5
    \setlength{\cellwidth}{46mm}
    \Tile{C5}{(B5.north east)}{\cellwidth}{\rowheight}
        {
            \titlebf{$1 < n < 2$} \\
            \vspace*{4mm}
            \includegraphics[width=39.5mm]{images/konkave_ecke.pdf} \\
            \vspace*{3mm}
            Konkave Ecke \\[3.0ex]
            $\left|w\right|\sim r^{n-1}$ \\[3.0ex]
            $r = 0 \Rightarrow$ Staupunkt
        }{0}{1}{1}{0}

    % D5) Vierte Kachel relativ zu C5
    \setlength{\cellwidth}{57mm}
    \Tile{D5}{(C5.north east)}{\cellwidth}{\rowheight}
        {
            \titlebf{$n > 2$} \\
            \vspace*{5mm}
            \includegraphics[width=49mm]{images/scharfe_konkave_ecke.pdf} \\
        }{0}{0}{1}{0}

% --- Sechse REIHE: unter A5, dann rechts anbauen
\setlength{\rowheight}{55mm}

    % A6) Erste Kachel relativ zu A5
    \setlength{\cellwidth}{78mm}
    \Tile{A6}{(A5.south west)}{\cellwidth}{\rowheight}
        {
            \titlebf{$n = \sfrac{1}{2}$} \\[-3.0ex]
            \begin{equation*}
                F(z) = 2\cdot a \cdot \sqrt{z} = 2\cdot a \cdot \sqrt{r}\cdot \left( \cos\left(\frac{\varphi}{2}\right) + i \sin\left(\frac{\varphi}{2}\right) \right)
            \end{equation*}
            \includegraphics[height=28mm]{images/ebene-randumströmung.pdf} \\[1.0ex]
            Ebene Randumströmung
        }{0}{1}{1}{0}

    % B6) Zweite Kachel relativ zu A6
    \setlength{\cellwidth}{56mm}
    \Tile{B6}{(A6.north east)}{\cellwidth}{\rowheight}
        {
            \titlebf{$n = 1$} \\[-3.0ex]
            \begin{equation*}
                F(z) = a \cdot z
            \end{equation*}
            \includegraphics[height=28mm]{images/ebene-translationsströmung.pdf} \\[2.5ex]
            Ebene Translationsströmung
        }{0}{1}{1}{0}

    % C6) Dritte Kachel relativ zu B6
    \setlength{\cellwidth}{56mm}
    \Tile{C6}{(B6.north east)}{\cellwidth}{\rowheight}
        {
            \titlebf{$n = 2$} \\[-3.0ex]
            \begin{equation*}
                F(z) = \frac{a}{2} \cdot z^2
            \end{equation*}
            \includegraphics[height=28mm]{images/ebene-staupunktströmung.pdf} \\[1.5ex]
            Ebene Staupunktströmung
        }{0}{0}{1}{0}

% --- Siebte REIHE: unter A6, dann rechts anbauen
\setlength{\rowheight}{45mm}

    % A7) Erste Kachel relativ zu A6
    \setlength{\cellwidth}{190mm}
    \Tile{A7}{(A6.south west)}{\cellwidth}{\rowheight}
        {
            \titlebf{Strömungsbild bestimmen} \\[3.0ex]

            \begin{tikzpicture}[node distance=6mm, >=Stealth]
                \node[draw, rounded corners, minimum width=1cm, minimum height=8mm] (A) {
                    $\varPsi(x,y)$
                };
                \node[draw, rounded corners, minimum width=1cm, minimum height=8mm, right=of A] (B) {
                    $u(x,y), v(x,y)$
                };
                \node[draw, rounded corners, minimum width=1cm, minimum height=8mm, right=of B] (C) {
                $
                    \left.
                    \begin{aligned}
                        u &= 0 \\
                        v &= 0
                    \end{aligned}
                    \ \right\}\Rightarrow\text{ Staupunkte }x_{0,i}\text{, }y_{0,i}
                $
                };
                \node[draw, rounded corners, minimum width=1cm, minimum height=8mm, right=of C] (D) {
                    $\varPsi_{0,i} = \varPsi\left(x_{0,i}\text{, }y_{0,i}\right)$
                };
                \node[draw, rounded corners, minimum width=1cm, minimum height=8mm, right=of D] (E) {
                    $\varPsi_{\text{S},i}\left(x,y\right) = \varPsi_{0,i}$
                };
                \draw[->] (A) -- (B);
                \draw[->] (B) -- (C);
                \draw[->] (C) -- (D);
                \draw[->] (D) -- (E);
            \end{tikzpicture}
        }{0}{0}{0}{0}
\end{tikzpicture}

% ### Zweite Seite der Potentialtheorie ###
\newpage
\begin{center}
    \section*{\underline{Potentialtheorie}}
    \titlebf{Singularitätentabelle}
\end{center}

\vspace*{-4mm}

\begin{table}[H]
    \centering
    \newlength{\rowhght}
    \setlength{\rowhght}{6.5mm}
    \footnotesize
    \renewcommand{\arraystretch}{2.0}
    \resizebox{\textwidth-20mm}{!}{\begin{tabularx}{205mm}{|l|c|c|c|c|Z|Z|Z|c|}
        \cline{1-9}
        \multirow{4}{*}{Bezeichnung} & \multirow{4}{*}{Stromlinienbild} & \multirow{2}{*}{\begin{tabular}[c]{@{}c@{}}\scriptsize Komplexe\\[-2.6ex]\scriptsize Potentialfunktion\end{tabular}} & \multicolumn{2}{c|}{Skalare} & \multicolumn{4}{c|}{\multirow{2}{*}{Geschwindigkeitskomponenten}} \\ 
        \cline{4-5}
         &  &  & \scriptsize Potentialfunktion & \scriptsize Stromfunktion & \multicolumn{4}{c|}{} \\ 
        \cline{3-9}
         &  & $F(z)$ & $\varPhi(x,y)$ & $\varPsi(x,y)$ & $u(x,y)$ & $v(x,y)$ & $v_r(x,y)$ & $v_\varphi(x,y)$ \\ 
        \cline{3-9}
         &  & $F(r,\varphi)$ & $\varPhi(r,\varphi)$ & $\varPsi(r,\varphi)$ & $u(r,\varphi)$ & $v(r,\varphi)$ & $v_r(r,\varphi)$ & $v_\varphi(r,\varphi)$ \\ 
        \Cline{1-9}{1.2pt}
        \xrowht[()]{\rowhght}\multirow{3}{*}{\begin{tabular}[l]{@{}l@{}}\\[\dimexpr(\rowhght/2)-10.8ex]Translations-\\[-2.6ex]Strömung in\\[-2.6ex]$x$-Richtung\end{tabular}} & \multirow{2}{*}{} & $u_\infty\,z$ & $u_\infty\,x$ & $u_\infty\,y$ & $u_\infty$ & $0$ & $u_\infty\,\tfrac{x}{\sqrt{x^2+y^2}}$ & $-u_\infty\,\tfrac{y}{\sqrt{x^2+y^2}}$ \\
        \cline{3-9}
        \xrowht[()]{\rowhght} &  & $u_\infty\,r\,e^{i\varphi}$ & $u_\infty\,r\,\cos(\varphi)$ & $u_\infty\,r\,\sin(\varphi)$ & $u_\infty$ & $0$ & $u_\infty\,\cos(\varphi)$ & $-u_\infty\,\sin(\varphi)$ \\ 
        \Cline{1-9}{1.2pt}
        \xrowht[()]{\rowhght}\multirow{3}{*}{\begin{tabular}[l]{@{}l@{}}\\[\dimexpr(\rowhght/2)-10.8ex]Translations-\\[-2.6ex]Strömung in\\[-2.6ex]$y$-Richtung\end{tabular}} & \multirow{2}{*}{} & $-i\,v_\infty\,z$ & $v_\infty\,y$ & $-v_\infty\,x$ & $0$ & $v_\infty$ & $v_\infty\,\tfrac{y}{\sqrt{x^2+y^2}}$ & $v_\infty\,\tfrac{x}{\sqrt{x^2+y^2}}$ \\ 
        \cline{3-9}
        \xrowht[()]{\rowhght} &  & $-i\,v_\infty\,r\,e^{i\varphi}$ & $v_\infty\,r\,\sin(\varphi)$ & $-v_\infty\,r\,\cos(\varphi)$ & $0$ & $v_\infty$ & $v_\infty\,\sin(\varphi)$ & $v_\infty\,\cos(\varphi)$ \\ 
        \Cline{1-9}{1.2pt}
        \xrowht[()]{\rowhght}\multirow{3}{*}{\begin{tabular}[l]{@{}l@{}}\\[\dimexpr(\rowhght/2)-10.8ex]Staupunkt-\\[-2.6ex]\& Eckenstr.\\[-2.6ex]$a\in\mathbb{R}^+$\end{tabular}} & \multirow{2}{*}{} & $\tfrac{a}{2}\,z^2$ & $\tfrac{a}{2}\left(x^2-y^2\right)$ & $a\,x\,y$ & $a\,x$ & $-a\,y$ & $a\,\tfrac{x^2-y^2}{\sqrt{x^2+y^2}}$ & $-a\,\tfrac{2\,x\,y}{\sqrt{x^2+y^2}}$ \\
        \cline{3-9}
        \xrowht[()]{\rowhght} &  & $\tfrac{a}{2}\,r^2\,e^{2i\varphi}$ & $\tfrac{a}{2}\,r^2\,\cos(2\,\varphi)$ & $\tfrac{a}{2}\,r^2\,\sin(2\,\varphi)$ & $a\,r\,\cos(\varphi)$ & $-a\,r\,\sin(\varphi)$ & $a\,r\,\cos(2\,\varphi)$ & $-a\,r\,\sin(2\,\varphi)$ \\
        \Cline{1-9}{1.2pt}
        \xrowht[()]{\rowhght}\multirow{3}{*}{\begin{tabular}[l]{@{}l@{}}\\[\dimexpr(\rowhght/2)-10.8ex]Quelle, Senke\\[-2.6ex](Ergiebigkeit\\[-2.6ex]sei $Q \neq 0$)\end{tabular}} & \multirow{2}{*}{} & $a$ & $a$ & $a$ & $a$ & $a$ & $a$ & $a$ \\
        \cline{3-9}
        \xrowht[()]{\rowhght} &   & $a$ & $a$ & $a$ & $a$ & $a$ & $a$ & $a$ \\
        \Cline{1-9}{1.2pt}
        \xrowht[()]{\rowhght}\multirow{3}{*}{\begin{tabular}[l]{@{}l@{}}\\[\dimexpr(\rowhght/2)-10.8ex]\scriptsize Potentialwirbel\\[-2.6ex](Zirkulation\\[-2.6ex]sei $\Gamma\neq 0$)\end{tabular}} & \multirow{2}{*}{} & $a$ & $a$ & $a$ & $a$ & $a$ & $a$ & $a$ \\
        \cline{3-9}
        \xrowht[()]{\rowhght} &   & $a$ & $a$ & $a$ & $a$ & $a$ & $a$ & $a$ \\
        \Cline{1-9}{1.2pt}
        \xrowht[()]{\rowhght}\multirow{3}{*}{\begin{tabular}[l]{@{}l@{}}\\[\dimexpr(\rowhght/2)-10.8ex]Dipol\\[-2.6ex](Auf der\\[-2.6ex]$x$-Achse)\end{tabular}} & \multirow{2}{*}{} & $a$ & $a$ & $a$ & $a$ & $a$ & $a$ & $a$ \\
        \cline{3-9}
        \xrowht[()]{\rowhght} &   & $a$ & $a$ & $a$ & $a$ & $a$ & $a$ & $a$ \\
        \Cline{1-9}{1.2pt}
        \xrowht[()]{\rowhght}\multirow{3}{*}{\begin{tabular}[l]{@{}l@{}}\\[\dimexpr(\rowhght/2)-10.8ex]Dipol\\[-2.6ex](Auf der\\[-2.6ex]$y$-Achse)\end{tabular}} & \multirow{2}{*}{} & $a$ & $a$ & $a$ & $a$ & $a$ & $a$ & $a$ \\
        \cline{3-9}
        \xrowht[()]{\rowhght} &   & $a$ & $a$ & $a$ & $a$ & $a$ & $a$ & $a$ \\
        \cline{1-9}
    \end{tabularx}}
\end{table}

\end{document}
